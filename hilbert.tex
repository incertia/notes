\section{Moving Towards Concreteness}
A lot of the things here will be restricted to the fields $\mathbb{F}
\in \lbrace \mathbb{C}, \mathbb{R}, \overline{\mathbb{Q}} \rbrace $.

\subsection{I'm In Space}
Because we needed more spaces in linear algebra.

\begin{df}
A \textbf{inner product space} $V$ is a vector space with an additional
structure called the \textbf{inner product} $\langle \cdot, \cdot
\rangle : V \times V \rightarrow \mathbb{F}$ satisfying
\begin{itemize}
\item \textbf{Conjugate symmetry} or \textbf{Hermitian symmetry}:
$\langle x, y \rangle = \overline{\langle y, x \rangle}$.
\item Linearity in the first argument: $\langle ax, y \rangle = a
\langle x, y \rangle$ and $\langle x + y, z \rangle = \langle x, z
\rangle + \langle y, z \rangle$.
\item \textbf{Positive definiteness}: $\langle x, x \rangle \geq 0$ and
$\langle x, x \rangle = 0 \implies x = 0$.
\end{itemize}
\end{df}

\begin{rem}
Because of conjugate symmetry, we have
\begin{enumerate}
\item $\langle x, cy \rangle = \overline{\langle cy, x \rangle} =
\overline{c} \overline{\langle y, x \rangle} = \overline{c} \langle x, y
\rangle$.
\item $\langle x, y + z \rangle = \overline{\langle y + z, x \rangle} =
\overline{\langle y, x \rangle} + \overline{\langle z, x \rangle} =
\langle x, y \rangle + \langle x, z \rangle$.
\end{enumerate}
\end{rem}

\begin{df}
A \textbf{metric space} is a set $M$ in which there is a metric
(distance) $d : M \times M \rightarrow \mathbb{R}$ such that
\begin{itemize}
\item $d(x, y) \geq 0$
\item $d(x, y) = 0 \Leftrightarrow x = y$
\item $d(x, y) = d(y, x)$
\item $d(x, z) \leq d(x, y) + d(y, z)$
\end{itemize}
\end{df}

\begin{df}
A \textbf{Cauchy sequence} is an infinite sequence that has a limit with
respect a consistent norm.
\end{df}

\begin{df}
A metric space is called \textbf{complete} if every Cauchy sequence in
$M$ has its limit in $M$.
\end{df}

\begin{df}
A \textbf{Hilbert space} is an inner product space that is also a
complete metrix space given by the norm $d(x, y) = \sqrt{\langle x - y,
x - y \rangle}$.
\end{df}

\begin{prb}
Verify that the triangle inequality holds for the above norm.
\end{prb}

\begin{proof}[Solution]
\begin{lem}[Cauchy-Schwarz]
$|\langle x, y \rangle | \leq \sqrt{\langle x, x \rangle} \sqrt{\langle y,
y \rangle}$.
\end{lem}
\begin{proof}
We have $|\langle x, y \rangle | = \sqrt{\langle x, y \rangle \langle y,
x \rangle}$, so we square to get $\langle x, y \rangle \langle y, x
\rangle \leq \langle x, x \rangle \langle y, y \rangle$, or $\langle x,
x \rangle \langle y, y \rangle - \langle x, y \rangle \langle y, x
\rangle \geq 0$. The inequality is clearly true when $x = 0$, so we
ignore that case and divide through by $\langle x, x \rangle$ to get
$\langle y, y \rangle - c \langle x, y \rangle = \langle y - cx, y
\rangle \geq 0$ where $c = \frac{\langle y, x \rangle}{\langle x, x
\rangle}$.

We can expand $\langle y - cx, y - cx \rangle = \langle y - cx, y
\rangle - \overline{c} \langle y - cx, x \rangle$, but $\langle y - cx,
x \rangle = \langle y, x \rangle - c \langle x, x \rangle = \langle y, x
\rangle - \langle y, x \rangle = 0$, so we get
\[ \langle y - cx, y \rangle = \langle y - cx, y - cx \rangle \geq 0, \]
as desired.

\end{proof}
We wish to show
\[ \sqrt{\langle x + y, x + y \rangle} \leq \sqrt{\langle x, x \rangle}
+ \sqrt{\langle y, y \rangle}. \]
Squaring gives
\[ \langle x, x \rangle + \langle x, y \rangle + \langle y, x \rangle +
\langle y, y \rangle \leq \langle x, x \rangle + \langle y, y \rangle +
2 \sqrt{\langle x, x \rangle \langle y, y \rangle}, \]
or
\[ \sqrt{\langle x, x \rangle} \sqrt{\langle y, y \rangle} +
\sqrt{\langle y, y \rangle} \sqrt{\langle x, x \rangle} \leq 2
\sqrt{\langle x, x \rangle \langle y, y \rangle} \]
by Cauchy-Schwarz, so we are done.
\end{proof}

\begin{df}
The \textbf{adjoint} of a matrix $L$ is its conjugate transpose, or
$L^{*} = \overline{L}^T$.
\end{df}

\begin{df}
We say a matrix $L$ is \textbf{self-adjoint} if $L = L^{*}$.
\end{df}

\begin{rem}
The eigenvalues of self-adjoint matrices are real.
\end{rem}
