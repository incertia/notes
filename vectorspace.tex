\section{Vector Spaces and Related Things}
We can finally start getting into this linear algebra thing.
\subsection{Vector Spaces}
We should now have enough abstract definitions to introduce vector
spaces in an abstract enough way. I am purposefully trying to stay away
from commutative maps because they confuse me and learning linear
algebra at such an abstract level is confusing enough for me.

\begin{df}
A \textbf{vector space} $V$ over a field $F$ is an abelian group under
$+$ and an $F$-module.
\end{df}

\begin{prop}
$0 = 0v$.
\end{prop}

\begin{proof}
$0v + 0v = (0 + 0) v = 0v$.
\end{proof}

\begin{prop}
$-v = -1 \cdot v$, or the additive inverse of $v$ is the additive
inverse in $F$ times $v$.
\end{prop}

\begin{proof}
$-1 \cdot v + v = -1 \cdot v + 1 \cdot v (-1 + 1) \cdot v = 0 \cdot v =
0$. This follows from the $F$-module structure that is imposed upon $V$,
where $1v = v$.
\end{proof}

\begin{df}
The \textbf{span} of a set of vectors $v_1, v_2, \dots, v_k$ is defined
as the set
\[ \left\lbrace \sum_i a_i v_i : a_i \in F \right\rbrace. \]
Similarly, a set of vectors span a set if every element of that set can
be written as a linear combination of the set of vectors.
\end{df}

\begin{df}
A set of vectors $v_1, \dots, v_k$ are said to be \textbf{linearly
independent} if $\sum_i a_i v_i = 0 \Leftrightarrow a_i = 0 \quad
\forall i$.
\end{df}

\begin{rem}
Notice that this guarantees that each decompisition into the linearly
independent vectors is unique.
\end{rem}

We now prove a fundamental lemma regarding linearly independent,
spanning sets.
\begin{lem}[Steinitz Exchange Lemma]
Let $V$ be a vector space. Let $v_1, v_2, \dots, v_k$ be a set of
linearly independent vectors in $V$. Let $w_1, w_2, \dots, w_m$ span
$V$. We then have $k \leq m$ and (possibly after reordering $w_i$), the
set of vectors $v_1, \dots, v_k, w_{k + 1}, \dots, w_m$ also span $V$.
\end{lem}

\begin{proof}
We induct on $k$ to get our result. The case $k = 0$ is obvious, as we
don't need to replace anything and the statement of the lemma solves
itself. So suppose the lemma holds positive for some $k < m$. Then,
possibly after reordering, the vectors $v_1, v_2, \dots, v_k, w_{k + 1},
\dots, w_m$ span $V$. We can then write
\[ v_{k + 1} = \sum_i \alpha_i v_i + \sum_i \alpha_i w_i. \]
At least one $\alpha_j$ must be non-zero, so we immediately get $k < m$,
or $k + 1 \leq m$. We can now ``solve'' for $w_{k + 1}$ to get
\[ w_{k + 1} = \frac{1}{\alpha_{k + 1}} \left(v_{k + 1} - \sum_{i =
1}^{k} \alpha_i v_i - \sum_{i = k + 2}^{m} \alpha_i w_i\right). \]
Hence $w_{k + 1}$ is in the span of the set $\lbrace v_1, \dots, v_{k +
1}, w_{k + 2}, \dots, w_m \rbrace$, which means that everything in the
span of $\lbrace v_1, \dots, v_k, w_{k + 1}, \dots, w_m \rbrace$ is also
in the span of $\lbrace v_1, \dots, v_{k + 1}, w_{k + 2}, w_m \rbrace$,
finishing the proof.
\end{proof}

\begin{df}
The cardinality of a set of linearly independent vectors that span a
vector space $V$ is called the \textbf{dimension} of $V$, denoted
$\dim(V)$.

This is also the cardinality of the largest set of linearly independent
vectors in $V$. This is easily seen by the above lemma.
\end{df}

\begin{df}
We say a set of vectors is a \textbf{basis} for a vector space $V$ if
the set contains linearly independent vectors and spans $V$.
\end{df}

\begin{df}
We say $W$ is a \textbf{subspace} of $V$ over $F$ if and only if $W$
satisfies:
\begin{itemize}
\item $0 \in W$
\item $w_1 + w_2 \in W \quad \forall w_1, w_2 \in W$
\item $k w \in W \quad \forall w \in W, k \in F$.
\end{itemize}
\end{df}

Anyways, here are some examples of vector spaces and subspaces
\begin{itemize}
\item Any field $F$, such as $\mathbb{R}$ or $\mathbb{C}$
\item Any repeated direct sum of a field $F$, such as $\mathbb{R}^3$
\item The space of functions on $\mathbb{R}$
\begin{itemize}
\item Odd functions on $\mathbb{R}$
\item Even functions on $\mathbb{R}$
\end{itemize}
\end{itemize}

\begin{prb}
Let $V$ be the space of functions $f : \mathbb{R} \rightarrow
\mathbb{R}$. Show that $V$ is isomorphic to $V_e \oplus V_o$, the space
of even and odd functions.
\end{prb}

\begin{proof}[Solution]
Let $f \in V$. Then $f = \frac{f(x) + f(-x)}{2} + \frac{f(x) -
f(-x)}{2} \cong \left(\frac{f(x) + f(-x)}{2}, \frac{f(x) -
f(-x)}{2}\right)$ and isomorphism follows from this decomposition in the
canonical fashion.
\end{proof}

\begin{prop}
$e^{\lambda x}$ and $e^{\mu x}$ are linearly independent if $\lambda
\neq \mu$.
\end{prop}

\begin{proof}
We set
\[ \left\lbrace \begin{aligned}
c_1 e^{\lambda x} + c_2 e^{\mu x} &= 0 \\
c_1 \lambda e^{\lambda x} + c_2 \mu e^{\mu x} &= 0
\end{aligned} \right. \]
Take $x = 0$ to get
\[ \left\lbrace \begin{aligned}
c_1 + c_2 &= 0 \\
c_1 \lambda + c_2 \mu &= 0
\end{aligned} \right. \implies c_1 = c_2 = 0. \]
\end{proof}

\begin{prop}
Let $V, W$ be subspaces. Then
\[ \dim(V) + \dim(W) = \dim(V + W) + \dim(V \cap W) \]
where $V + W = \lbrace v + w : v \in V, w \in W \rbrace$.
\end{prop}

\begin{proof}
Let $\alpha_1, \dots, \alpha_n$ be a basis for $V \cap W$. We can then
find the basis $\alpha_1, \dots, \alpha_n, v_1, \dots, v_k$ for $V$ and
the same for $W$, $w_1, \dots, w_l, \alpha_1, \dots, \alpha_n$. If we
can prove that $\alpha_1, \dots, \alpha_n, v_1, \dots, v_k, w_1, \dots,
w_l$ are a basis for $V + W$, we are done because
\[ (n + k) + (n + l) = (n + k + l) + n \]
It is obvious that they span $V + W$ by the definition of span, so it
suffices to show that they are linearly independent. Suppose that
\[ 0 = \sum a_i \alpha_i + \sum b_i v_i + \sum c_i w_i \]
We rearrange this equation to arrive at
\[ -\sum a_i \alpha_i = \sum b_i v_i + \sum c_i w_i, \]
implying $-\sum a_i \alpha_i, \sum b_i v_i + \sum c_i w_i \in V \cap W$.
However, $v_i$ and $w_i$ are not in the set $V \cap W$, otherwise they
would not be linearly independent with the $\alpha_i$, which implies
that $b_i = c_i = 0$ for all $i$, otherwise the RHS would not be in the
set $V \cap W$. Now use the fact that $\alpha_i$ form a basis to get
that $a_i = 0$, so we are done.
\end{proof}

\begin{prop}
A vector space $V$ over a field $F$ with dimension $d$ is isomorphic to
$F^d$.
\end{prop}

\begin{proof}
We just choose a basis and express each vector in $V$ as the
coefficients in the expansion as a sum of linearly independent vectors.
\end{proof}

For example, consider the space of polynomials with complex coefficients
with degree less than or equal to $3$. We will call this space
$\mathcal{P}_3$. Then the vector $1 + (3 - 2i)x + 7x^2$ is the same as
$(1, 3 - 2i, 7, 0)$ with the basis of $\lbrace 1, x, x^2, x^3 \rbrace$.

\begin{prb}
Determine a necessary and sufficient condition for a polynomial $P \in
\mathbb{R}[X]$ such that $P(x) \in \mathbb{Z} \forall x \in
\mathbb{Z}$.
% TODO: Actually prove this thing
\end{prb}

\begin{proof}[Solution]
I claim that $P$ must be an integral linear combination of the
polynomials $\binom{X}{k} \forall k \in \mathbb{N}$.
\end{proof}

\subsection{Algebraic Manipulation}
Here we do various interesting things with subspaces.

\begin{prop}
Let $V, W$ be subspaces. Then $V \oplus W \cong V + W$ if $V \cap W =
\lbrace 0 \rbrace$. I'm not sure if this can be extended to only if.
\end{prop}

\begin{proof}
Take $\varphi(v, w) = v + w$, which is clearly bijective. We wish to
show that $\varphi^{-1}(v + w) = (v, w)$ is also bijective. Suppose $v_1
+ w_1 = v_2 + w_2$. Then $v_1 - v_2 = w_2 - w_1 = 0$, so $v_1 = v_2$ and
$w_1 = w_2$, completing the injectivity portion. Surjectivity is obvious
so we are done as the preservation of structure is trivial.
\end{proof}

\subsection{Linear Maps}
\begin{df}
Let $V, W$ be vector spaces.  A map $L : V \rightarrow W$ is said to be
\textbf{linear} if it satisfies:
\begin{itemize}
\item $L(v_1 + v_2) = L(v_1) + L(v_2) \quad \forall v_1, v_2 \in V$
\item $L(c v) = c L(v) \quad \forall c \in F, v \in V$
\end{itemize}
\end{df}

\begin{df}
The \textbf{nullspace} of a linear map $L$ is the same as the kernel of
$L$. We will denote the nullspace with $\mathcal{N}(L)$.
\end{df}

\begin{df}
The \textbf{range} of $L$ is simply the set $\mathcal{R}(L) = \lbrace
L(v) : v \in V \rbrace$.
\end{df}

\begin{prop}
Let $V, W$ be vector spaces and let $L : V \rightarrow W$ be linear.
Then $L$ is injective if and only if $\mathcal{N}(L) = \lbrace 0
\rbrace$.
\end{prop}

\begin{proof}
The forward direction is trivial because $L(v) = 0 = L(0) \implies v =
0$.

Going backwards, suppose $L(v_1) = L(v_2)$. Then, we have
\[ \begin{aligned}
L(v_1) - L(v_2) &= 0 \\
L(v_1 - v_2) &= 0 \\
v_1 - v_2 &= 0 \\
v_1 &= v_2
\end{aligned} \]
by linearity in $L$ and the fact that $v_1 - v_2 \in \mathcal{N}(L)$.
\end{proof}

\begin{thm}[Rank-Nullity Theorem]
Let $V, W$ be finite dimensional vector spaces and let $L : V
\rightarrow W$ be linear. Then
\[ \dim(\mathcal{N}(L)) + \dim(\mathcal{R}(L)) = \dim(V). \]
\end{thm}

\begin{proof}
Let $\lbrace \alpha_i \rbrace_{i = 1}^{k}$ be a basis for
$\mathcal{N}(L)$. By the exchange lemma, we also have $v_{k + 1}, \dots,
v_n$ so that $\alpha_i, v_i$ form a basis for $V$. If we can show that
$\dim(\mathcal{R}(L)) = n - k$, we are done, so we claim that $L(v_{k +
1}), \dots, L(v_n)$ form a basis for $\mathcal{R}(L)$.

\textbf{Span}: Suppose $z \in \mathcal{R}(L)$. Then we have
\[ \begin{aligned}
z &= L(v) \\
z &= L\left(\sum a_i \alpha_i + \sum b_i v_i\right) \\
z &= \sum a_i L(\alpha_i) + \sum b_i L(v_i) \\
z &= \sum b_i L(v_i) \implies z \in \operatorname{span}(L(v_i))
\end{aligned} \]
by linearity in $L$ and the stupid reason that $L(\alpha_i) = 0$.

\textbf{Linear independence}: Suppose $0 = \sum a_i L(v_i)$. Then
$L\left(\sum a_i v_i\right) = 0 \implies \sum a_i v_i \in \mathcal{N}(L)
\implies \sum a_i v_i = \sum b_i \alpha_i$, or $\sum a_i v_i + \sum -b_i
v_i = 0$, giving $a_i, b_i = 0$.
\end{proof}

\begin{rem}
$\dim(\mathcal{N}(L))$ is usually called the nullity of $L$ while
$\dim(\mathcal{R}(L))$ is usually called the rank of $L$, hence the name
of the theorem.
\end{rem}

\subsection{Matrices of Linear Maps}
Because linear maps are linear, they are given completely by what they
do on a basis. This motivates us to find some sort of compact
representation for a linear map.

\begin{df}
The \textbf{matrix} of a linear map $L : V \rightarrow W$ is defined as
follows. We choose a basis for $V$ and $W$, and the columns of the
matrix of $L$ is the coordinate decomposition of $L$ on the basis of
$V$.
\end{df}

\begin{ex}
The matrix of the linear transforomation in $\mathbb{R}^3$ to
$\mathbb{R}^3$ that takes $(1, 0, 0), (0, 1, 0), (0, 0, 1)$ to $(1, 2,
3),  (4, 5, 6), (7, 8, 9)$ would be represented as
\[ \begin{pmatrix}
1 & 4 & 7 \\
2 & 5 & 8 \\
3 & 6 & 9
\end{pmatrix} \]
\end{ex}

\begin{ex}
Consider the polynomial space $\mathcal{P}_3$ and let the linear
transformation be differentiation. The matrix of differentiation (taking
$\mathcal{P}_3$ to $\mathcal{P}_2$) with respect to $1, x, x^2, x^3$ and
$1, x, x^2$ would be
\[ \begin{pmatrix}
0 & 1 & 0 & 0 \\
0 & 0 & 2 & 0 \\
0 & 0 & 0 & 3
\end{pmatrix} \]
\end{ex}

We can then easily compute the transformation of any vector once bases
are chosen. This is done by ``dotting'', or multiplying each row
vector's components with the components of the column vector
representation of the vector being transformed, and then summing up the
multiplied components.

\begin{ex}
\[ \begin{pmatrix}
0 & 1 & 0 & 0 \\
0 & 0 & 2 & 0 \\
0 & 0 & 0 & 3
\end{pmatrix} \begin{pmatrix}
1 \\ 2 \\ 3 \\ 4
\end{pmatrix} = \begin{pmatrix}
2 \\ 6 \\ 12
\end{pmatrix}, \]
which is the same as saying $\frac{d}{dx}(1 + 2x + 3x^2 + 4x^3) = 2 + 6x
+ 12x^2$.
\end{ex}

\begin{rem}
We can compose linear transformations. If $T_1 : U \rightarrow V, T_2 :
V \rightarrow W$, then we can compute $T_3 : U \rightarrow W$ with $T_3
= T_2 T_1$, which is defined as $T_3(u) = T_2(T_1(u))$. Computing the
matrix of the resulting transform is also simple, and this is what
defines \textbf{matrix multiplication}. You basically transform each
column vector in the matrix of $T_1$ by $T_2$ and set that as the column
vector in the respective column of $T_3$. If this does not make sense,
experiment a little bit and see why this should be true.
\end{rem}

\begin{rem}
Matrix multiplication is associative in the finite dimensional case. The
easiest way to see this is by going back to composition of
transformations. $T_3 (T_2 T_1) = (T_3 T_2) T_1$ simply because $T_3
(T_2 T_1) u = T_3(T_2(T_1(u))) = (T_3 T_2) T_1 u$. We can also see this
by expanding out the $ij$-th entry in the resultant matrix and noticing
that only the order of summation is changed, which is not important in
finite dimensions.
\end{rem}

\begin{ex}
Here is a case where multiplication of matrices or composition of linear
operators is not associative. Consider the vector space $\mathbb{R}[X]$.
Denote by $Df = \frac{df}{dX}, f \in \mathbb{R}[X]$ and $If = \int_0^X
f(s) ds, f \in \mathbb{R}[X]$. Clearly $DIf = f$, but $IDf = f - f(0)$.
\end{ex}

\begin{df}
The \textbf{transpose} $A^T$ of a matrix $A$ is given by interchanging
the rows and columns of $A$.
\end{df}

\begin{rem}
$A_{ij}$ will denote the $i,j$-th entry of a matrix $A$, or the entry in
the $i$-th row and $j$-th column.
\end{rem}

\begin{df}
We will denote by $\mathcal{M}(m, n)$ the space of $m \times n$
matrices, or matrices with $m$ rows and $n$ columns with addition and
scalar multiplication done component wise.
\end{df}

\begin{rem}
Let $M = \left\lbrace \begin{pmatrix} a & -b \\ b & a \end{pmatrix} : a, b
\in \mathbb{R}\right\rbrace$. Then there is the bijection $\mathbb{C}
\leftrightarrow M$ given by $a + bi \leftrightarrow \begin{pmatrix} a &
-b \\ b & a \end{pmatrix}$.
\end{rem}
