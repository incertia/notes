\section{Day 1 - Math Reboot}

\subsection{Discussion Review}

\begin{prb}
From discussion (P4), given $\vec{b} \cdot \vec{v} = \lambda$ and $\vec{b}
\times \vec{v} = \vec{c}$, what is $\vec{v}$ in terms of $\lambda,
\vec{b}, \vec{c}$.
\end{prb}

\begin{proof}[Solution]
Given orthogonal vectors $\vec{e_b}, \pvec{e_b}'$, we write $\vec{v} = a
\vec{e_b} + b \pvec{e_b}'$ where $a$ is the length along $\vec{b}$ and
$b$ is the length perpendicular to $\vec{b}$. However, $a = |\vec{v}|
\cos \theta = |\vec{v}| \frac{\vec{v} \cdot \vec{b}}{|\vec{v}|
|\vec{b}|} = \frac{\lambda}{|\vec{b}|}$ and $b = |\vec{v}| \sin \theta =
|\vec{v}| \frac{|\vec{c}|}{|\vec{v}| |\vec{b}|} =
\frac{|\vec{c}|}{|\vec{b}|}$. Now take $\vec{e_b} = \frac{1}{|\vec{b}|}
\vec{b} = x \vec{e_x} + y \vec{e_y}$ and $\pvec{e_b}' = y \vec{e_x} - x
\vec{e_y}$ where $\vec{e_x}, \vec{e_y}$ are arbitrary choices of
orthogonal unit vectors lying in the plane containing $\vec{v}, \vec{b}$
(usually $(1, 0), (0, 1)$, respectively).
\end{proof}

\subsection{Differential Equations}

How to practically solve differential equations in this class.

For $n$-th order ODEs...
\begin{enumerate}
\item Separation of variables
\item Guess, and then chug and plug (guess must have $\geq n$
parameters)
\item Use Wolfram{\textbar}Alpha
\end{enumerate}

\begin{prb}
Consider a block of mass $m$ sitting on an inclined plane of angle
$\theta$ with standard gravity $\vec{g}$. The plane has frictional
coefficient $\mu$. Given that the block starts at rest, determine the
position of the block at arbitrary time $t$.

\begin{center}
\begin{asy}
size(6cm);

pair O = (0, 0);
pair B = (0, 1);
pair C = (3, 0);

path p = B -- C;
pair p1 = point(p, 0.1);
pair p2 = point(p, 0.23);
pair n = 2 * (1, 3) / 25;
pair m = point(p1 -- p2, 0.5);

draw(O -- B -- C -- cycle);
draw(p1 -- p1 + n -- p2 + n -- p2);

draw("$\theta$", arc(C, 0.5, 162, 180), left + (0, 0.1));
draw("$g$", m -- m + (0, -0.5), Arrow);

label("$m$", m + n / 2);
label("$\mu$", point(p, 0.5), N, blue);
\end{asy}
\end{center}
\end{prb}

\begin{proof}[Solution]
We refer back to the magical formula
\[ \vec{F} = m\ddot{\vec{r}}. \]
$\ddot{\vec{r}}$ is trivially $\ddot{x} \vec{e}_x + \ddot{y} \vec{e}_y$.

Luckily we have the relations
\[ \begin{aligned}
\vec{F}_g &= m\vec{g} = mg(\vec{e}_x \sin \theta - \vec{e}_y \cos
\theta) \\
\vec{F}_f &= -\mu |\vec{F}_N| \vec{e}_x \\
\vec{F}_N &= -\proj_{\vec{e}_y} \vec{F}_g = -mg\cos\theta \vec{e}_y, \\
\end{aligned} \]
so basically we have $F_x = mg\sin\theta - \mu mg \cos \theta =
m\ddot{r}_x$, or $\ddot{r}_x = c \implies r_x = c_1 t^2 + c_2 t + c_3$
and solving for the constants is trivial.
\end{proof}
