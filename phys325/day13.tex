\section{October 6, 2015}

\subsection{Total KE}
We present another way to to solve the wheel rolling down a ramp.

We have that $T + U$ is some wheel invariant and compute
\[ \begin{aligned}
U &= \sum_i m_i g h_i \\
&= g \sum_i m_i h_i \\
&= g M H_{CM} = Mg(-X\sin\theta). \\
\end{aligned} \]
We also know that we have
\[ \begin{aligned}
T &= T_{CM} + T' \\
&= \frac{1}{2} M \dot{X}^2 + \frac{1}{2} I \omega^2. \\
\end{aligned} \]
After using the fact that the wheel is rolling without slipping, or
$\dot{X} = b \omega$, we get
\[ T + U = \frac{1}{2}\left(M \dot{X}^2 + I \frac{\dot{X}^2}{b^2}\right)
- MgX\sin\theta = C. \]
Taking the derivative, we arrive at
\[ 0 = \frac{d}{dt}(T + U) = \dot{X} \ddot{X}\left(M +
\frac{I'}{b^2}\right) - \dot{X}Mg\sin\theta, \]
which implies that
\[ \ddot{X} = \frac{Mg\sin\theta}{M + \dfrac{I'}{b^2}}. \]

\subsection{Total KE Again}
We can do this in another similar way by considering the energy relative
to the contact point because
\[ T = T^{(B)} = \frac{1}{2}I^{(B)} \omega^2 = \frac{1}{2}(I_{CM}^{(B)}
+ I')\omega^2 = \frac{1}{2}(Mb^2 + I')\omega^2, \]
which, after staring long enough, is equivalent to $T + T'$ in the
previous part.

\subsection{Conservation of Things}
$T + U$ is conserved, but under what conditions?

Recall that $U(\vec{r})$ is a scalar field representation of a force
field with
\[ \vec{F} = - \vec{\nabla}U \iff U(\vec{r}) = -
\int_{\vec{r}_0}^{\vec{r}} \vec{F} \cdot d\vec{r} \]

\begin{prop}
$T + U$ is conserved when all forces are considered conservative or ``no
work'' forces.
\end{prop}

\begin{df}
A force is \textbf{conservative} if
\begin{enumerate}
\item It is at most position dependent.
\item It is irrotational, or $\vec{F} = -\vec{\nabla}U$.
\end{enumerate}
\end{df}

\begin{proof}
Compute
\[ \begin{aligned}
dT &= d\left(\frac{1}{2} mv^2\right) \\
&= \frac{m}{2}d(\vec{v} \cdot \vec{v}) \\
&= \frac{m}{2}(d\vec{v} \cdot \vec{v} + \vec{v} \cdot d\vec{v}) \\
&= md\vec{v} \cdot \vec{v} = d\vec{p} \cdot \frac{d\vec{r}}{dt} \\
&= \vec{F} \cdot d\vec{r} = -\vec{\nabla}U \cdot d\vec{r} \\
&= -\left(\frac{\partial U}{\partial x}dx + \frac{\partial U}{\partial
y}dy + \frac{\partial U}{\partial z}dz \right) \\
\end{aligned} \]
and
\[ \begin{aligned}
dU &= \frac{\partial U}{\partial x}dx + \frac{\partial U}{\partial y}dy
+ \frac{\partial U}{\partial z}dz + \gamma \\
\end{aligned} \]
where $\gamma$ represents the other dependencies of $U$.  Now $dT + dU =
\gamma$, so it is conserved only if there are no other dependencies.
\end{proof}

\begin{thm}[Irrotational Field Theorem]
These statements are equivalent.
\begin{enumerate}
\item $\vec{F} = \vec{\nabla}g$ for some $g$.
\item $\vec{\nabla} \times \vec{F} = \vec{0}$.
\item $\oint \vec{F} \cdot d\vec{r} = 0$ around any closed loop.
\item $\int_{\vec{a}}^{\vec{b}} \vec{F} \cdot d\vec{r}$ is path
independent.
\end{enumerate}
\end{thm}
