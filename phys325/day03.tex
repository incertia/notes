\section{Day 3}

\subsection{Notes on Homework 1}
Learn how to do math. For example $1 - x^2 = (1 + x)(1 - x)$. Also trig
is nice.
\[ \begin{aligned}
\sin (A + B) &= \sin A \cos B + \cos A \sin B \\
\cos (A + B) &= \cos A \cos B + \sin A \sin B \\
\end{aligned} \]
Please use $\vec{F} = m \vec{a}$. Don't be stupid.

\subsection{Things}
Most of the time was spent covering the final example from
section~\ref{cylindrical}, which is kept in section~\ref{cylindrical}.

\subsection{2 Second Problems}
\begin{prb}[Triangle Inequality]
Show that $|\vec{a} + \vec{b}| \leq a + b$.
\end{prb}
\begin{proof}[Solution]
Expand.
\[ a^2 + b^2 + 2ab\cos\theta \leq a^2 + b^2 + 2ab. \]
\end{proof}

\begin{prb}
Evaluate
\[ \frac{d}{dt}\left(\vec{a} \cdot \left( \vec{v} \times \vec{r} \right)
\right). \]
\end{prb}
\begin{proof}[Solution]
Use product rule.
\[ \dot{\vec{a}} \cdot \left(\vec{v} \times \vec{r}\right)  + \vec{a}
\cdot \left(\vec{a} \times \vec{r}\right) + \vec{a} \cdot \left(\vec{v}
\times \vec{v}\right) = \dot{\vec{a}} \cdot \left(\vec{v} \times \vec{r}
\right). \]
\end{proof}

\begin{prb}
Consider $\vec{F} = F(x) \hat{x}$. Solve $F(x) = ma = m \ddot{x}$.
\end{prb}
\begin{proof}[Solution]
Write
\[ \begin{aligned}
F(x) &= ma = m \frac{dv}{dt} = m \frac{dx}{dt} \frac{dv}{dx} \\
\implies F(x) dx &= mv dv \\
\int_{x_0}^x F(u) du &= \frac{1}{2} m(v^2 - v_0^2) \\
v &= \sqrt{\frac{2}m \int_{x_0}^x F(u) du + v_0^2} \\
x &= \int_{t_0}^t \sqrt{\frac{2}m \int_{x_0}^x F(u) du + v_0^2} ds \\
\end{aligned} \]
and pray.
\end{proof}

\begin{prb}
Consider a bead located on a circular hoop of radius $R$ aligned with
the $z$ axis rotating with angular velocity $\omega$.
\end{prb}

\begin{proof}[Solution]
Here we have $\phi = \omega t + \phi_0, \dot{\phi} = \omega$. We have
$\vec{F}_g = -g\hat{z}, \vec{F}_s = -mR \omega^2 \hat{s}, \vec{F}_\theta
= -mR \dot{\theta}^2 \hat{r}$.

We also have $F_\theta = \sin \theta (r \ddot{\phi} + 2 \dot{r}
\dot{\phi}) + \cos \theta (2 r \dot{\theta} \dot{\phi})$, but
$\ddot{\phi} = 0, \dot{r} = 0$ so $F_\theta = 2 R \omega \dot{\theta}
\cos \theta$.
\end{proof}
