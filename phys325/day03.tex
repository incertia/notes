\section{Day 3}

\subsection{Notes on Homework 1}
Learn how to do math. For example $1 - x^2 = (1 + x)(1 - x)$. Also trig
is nice.
\[ \begin{aligned}
\sin (A + B) &= \sin A \cos B + \cos A \sin B \\
\cos (A + B) &= \cos A \cos B + \sin A \sin B \\
\end{aligned} \]
Please use $\vec{F} = m \vec{a}$. Don't be stupid.

\subsection{Things}
Most of the time was spent covering the final example from
section~\ref{cylindrical}, which is kept in section~\ref{cylindrical}.

\subsection{2 Second Problems}
\begin{prb}[Triangle Inequality]
Show that $|\vec{a} + \vec{b}| \leq a + b$.
\end{prb}
\begin{proof}[Solution]
Expand.
\[ a^2 + b^2 + 2ab\cos\theta \leq a^2 + b^2 + 2ab. \]
\end{proof}

\begin{prb}
Evaluate
\[ \frac{d}{dt}\left(\vec{a} \cdot \left( \vec{v} \times \vec{r} \right)
\right). \]
\end{prb}
\begin{proof}[Solution]
Use product rule.
\[ \dot{\vec{a}} \cdot \left(\vec{v} \times \vec{r}\right)  + \vec{a}
\cdot \left(\vec{a} \times \vec{r}\right) + \vec{a} \cdot \left(\vec{v}
\times \vec{v}\right) = \dot{\vec{a}} \cdot \left(\vec{v} \times \vec{r}
\right). \]
\end{proof}

\begin{prb}
Consider $\vec{F} = F(x) \hat{x}$. Solve $F(x) = ma = m \ddot{x}$.
\end{prb}
\begin{proof}[Solution]
Write
\[ \begin{aligned}
F(x) &= ma = m \frac{dv}{dt} = m \frac{dx}{dt} \frac{dv}{dx} \\
\implies F(x) dx &= mv dv \\
\int_{x_0}^x F(u) du &= \frac{1}{2} m(v^2 - v_0^2) \\
v &= \sqrt{\frac{2}m \int_{x_0}^x F(u) du + v_0^2} \\
x &= \int_{t_0}^t \sqrt{\frac{2}m \int_{x_0}^x F(u) du + v_0^2} ds \\
\end{aligned} \]
and pray.
\end{proof}

\begin{prb}
\label{rotatinghoop}
Consider a bead located on a circular hoop of radius $R$ aligned with
the $z$ axis rotating with angular velocity $\omega$. Given normal
gravity $g$, find the general motion of the bead.
\end{prb}

\begin{proof}[Solution]
Here we have $\phi = \omega t + \phi_0, \dot{\phi} = \omega$. The
primary forces are $\vec{F}_g = -mg\hat{z} = -mg \cos \theta \hat{r} +
mg \sin \theta \hat{\theta}, \vec{F}_N = F_{N,r} \hat{r}
+ F_{N,\phi} \hat{\phi}$. $F_\theta$ on the LHS is equivalent to
$mg\sin \theta$.

We also have $F_\theta = m \vec{a} = m \left( r \ddot{\theta} + 2
\dot{r} \dot{\theta} - r \dot{\phi}^2 \sin \theta \cos \theta \right)$,
but $\dot{\phi} = \omega, \dot{r} = 0$ so $F_\theta = m R
\left(\ddot{\theta} - \omega^2 \sin \theta \cos \theta\right)$.

This implies that $\ddot{\theta} = \left(\frac{g}{R} + \omega^2 \cos
\theta \right) \sin \theta$. Doing some rearranging, we get
$\ddot{\theta} = \frac{d\dot{\theta}}{dt} \frac{d\theta}{d\theta} =
\frac{d\dot{\theta}}{d\theta} \dot{\theta}$, which gives us
\[ \left(\frac{g}{R} + \omega^2 \cos \theta\right) \sin \theta d \theta
= \dot{\theta} d \dot{\theta}, \]
which is solvable for $\dot{\theta}$ which gives $\theta$.
\end{proof}
