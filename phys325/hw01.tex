\documentclass{article}

\usepackage[margin=1in]{geometry}
\usepackage{amsmath}
\usepackage{amsthm}
\usepackage{amsfonts}
\usepackage{amssymb}
\usepackage{asymptote}
\usepackage{enumerate}
\usepackage{fancyhdr}
\usepackage{mathtools}
\usepackage{parskip}

% setup the header
\pagestyle{fancy}
\lhead{Will Song}
\chead{Phys 325}
\rhead{HW1}

% setup definition/theorem etc
\newtheoremstyle{norm}
{3pt}
{3pt}
{}
{}
{\bf}
{:}
{.5em}
{}

\theoremstyle{norm}
\newtheorem{thm}{Theorem}[section]
\newtheorem{lem}[thm]{Lemma}
\newtheorem{df}[thm]{Definition}
\newtheorem{rem}[thm]{Remark}
\newtheorem{st}{Step}
\newtheorem{prop}[thm]{Proposition}
\newtheorem{cor}[thm]{Corollary}
\newtheorem{conj}[thm]{Conjecture}
\newtheorem{clm}[thm]{Claim}
\newtheorem{exr}[thm]{Exercise}
\newtheorem{ex}[thm]{Example}
\newtheorem{prb}[thm]{Problem}
\newtheorem{note}[thm]{Note}

% just useful shorthand
\renewcommand{\st}{\,\operatorname{s.t.}\,}
\let\hom\relax
\DeclareMathOperator{\hom}{Hom}
\DeclareMathOperator{\Tr}{Tr}
\DeclareMathOperator{\sgn}{sgn}
\DeclareMathOperator{\adj}{adj}
\DeclareMathOperator{\proj}{proj}
\newcommand{\pvec}[1]{\vec{#1}\mkern2mu\vphantom{#1}}

% for easier math
\everymath{\displaystyle}

\begin{document}

\section{Homework 1}

\begin{prb}[Hard Projectile Motion]
Consider a inclined plane angled at an angle of $\theta$ with respect to
the $xy$ plane. Call one corner touching the ground the origin and the
edge lying against the ground the $x$ axis. The $y$ axis runs along the
inclined edge of the inclined plane and the $z$ axis is normal to the
plane. A frictionless puck at rest is kicked so that it slides along the
plane with initial velocity $\vec{v}_0 = v_{0_x} \hat{x} + v_{0_y}
\hat{y}$ and normal gravity $g$. Compute how far the puck travles once
it returns to floor level.
\end{prb}

\begin{proof}[Solution]
We begin by drawing a very nice diagram.
\begin{center}
\begin{asy}
import three;
import graph3;
import solids;

settings.render = 0;
settings.prc = false;
settings.multisample = 1;

size(10cm);

currentprojection = perspective(30, 25, 8);

real z = 7;
real x = 25;
triple o = x * X + 24 Y;
path3 pl = z * Z -- x * X + z * Z -- o -- 24Y -- cycle;

// draw axes
draw(L = Label("$x'$", position = Relative(1.1), align = SW), O -- x*X,
red + 0.75, Arrow3);
draw(L = Label("$y'$", position = Relative(1.1), align = SE), O -- 28Y,
red + 0.75, Arrow3);
draw(L = Label("$z'$", position = Relative(1.1), align =  N), O --
(z+1)*Z, red + 0.75, Arrow3);

// draw plane
draw(surface(O -- x*X -- x*X + 24Y -- 24Y -- cycle), emissive(gray +
opacity(0.4)));
draw(surface(pl), emissive(gray + opacity(0.5)));

// compute new unit vectors
triple xp = unit(24Y - o);
triple yp = unit(x*X + z*Z - o);
triple zp = cross(xp, yp);

// draw new coordinates
draw(L = Label("$x$", position = Relative(1.1), align = S), o -- o + 6
xp, blue + 0.75, Arrow3);
draw(L = Label("$y$", position = Relative(1.1), align = W), o -- o + 6
yp, blue + 0.75, Arrow3);
draw(L = Label("$z$", position = Relative(1.1), align = E), o -- o + 6
zp, blue + 0.75, Arrow3);
dot(L = Label("$O$", align = SW), o);

// draw initial vector
path3 p = o -- o + 6 xp + 8 yp;
triple dir = unit(6 xp + 8 yp);
draw(L = Label("$\vec{v}_0 = v_{0_x} \hat{x} + v_{0_y} \hat{y}$",
position = Relative(1.6), align = unit(NW + 2N)), p, Arrow3);

// mark angle to vector
triple pp = o + 2 dir;
triple xx = o + 2 xp;
triple yy = o - 2 Y;
triple yz = o + 2 yp;

draw(arc(o, pp, xx));
draw(L = Label("$\theta$", align = SW), arc(o, yz, yy));

label(Label("$m$", align = SE), o);

// draw gravity
draw(L = Label("$g$", position = Relative(0.5), align = E), o -- o -
5Z, Arrow3);
\end{asy}
\end{center}

Again we refer back to the magical law $\vec{F} = m \ddot{\vec{r}}$ and
realize this is fairly simple as the only force we experience here is
gravity and the normal force. In other words,
\[ \begin{aligned}
\vec{F} = -mg \hat{z}' + \vec{F}_N &= -mg(\proj_{\hat{x}} \hat{z}' +
\proj_{\hat{y}} \hat{z}' + \proj_{\hat{z}} \hat{z}') + \vec{F}_N \\
&= -mg\left( (\hat{y} \cdot \hat{g}) \hat{y} + k \hat{z} \right) +
\vec{F}_N \\
&= -mg \left( \sin \theta \hat{y} + k \hat{z} \right) + \vec{F}_N.
\end{aligned} \]

Luckily $\vec{F}_N$ is constrained along the $z$ axis such that $z = 0$
so we can disregard that and just look at the $y$ coordinate. There is
no acceleration in the $x$ direction so we just have $x = v_{0_x} t$.
However, the $y$ coordinate experiences constant acceleration $-mg \sin
\theta$ so we have $y = - \frac{g}{2} \sin \theta t^2 + v_{0_y} t$ which
reaches $0$ ($2 * \textrm{maximum}$ due to symmetry) at $t = \frac{2
v_{0_y}}{g \sin \theta}$ So the distance from the $x$ coordinate is just
\[ \boxed{\frac{2 v_{0_x} v_{0_y}}{g \sin \theta}}, \]
which is what we would expect from normal projectile motion.
\end{proof}

\begin{prb}
Consider an inclined plane angled at an angle $\phi$. A ball is thrown
up towards the plane with initial velocity $v_0$ at an angle of $\theta$
above the plane. Let the $x$ axis be pointing up the slope and the $y$
axis normal to the slope. Disregard the $z$ axis as there is no motion
in that direction. Compute

\begin{enumerate}[(a)]
\item The ball's position $\vec{r}(t)$. This needs to be valid whenever
the ball is above the plane.
\item Show that the ball lands at a distance $\frac{2 v_0^2 \sin \theta
\cos (\theta + \phi)}{g \cos^2 \phi}$ from the origin.
\item With fixed $v_0, \phi$, the maximum distance is at $\frac{v_0^2}{g
+ g \sin \phi}$.
\end{enumerate}
\end{prb}

\begin{proof}[Solution]
We begin with another detailed diagram. This time in $2$ dimensions as
there is nothing to worry about with respect to $z$.

\begin{center}
\begin{asy}
size(10cm);

pair O = (0, 0);
pair X = (1, 0);
pair Y = (0, 1);

pair A = O;
pair B = 7Y;
pair C = 24X;

draw(A -- B -- C -- cycle);

// get axes
pair xp = unit(B - C);
pair yp = rotate(-90) * xp;

// draw axes
draw(L = Label("$x$", position = Relative(1.0), align = N), C -- C + 7
xp, black + 0.75, Arrow);
draw(L = Label("$y$", position = Relative(1.0), align = N), C -- C + 7
yp, black + 0.75, Arrow);

// get a random vector
pair v = unit((-5, 12));

// draw the v and v_0
draw(L = Label("$v_0$", align = NE), C -- C + 8 v, Arrow);

label("$\vec{v}$", C + 8 v, N);

// mark phi
draw(L = Label("$\phi$", position = Relative(0.5), align = unit(40W +
6N)), arc(C, C + 3 * xp, C - 3 * X));

// mark theta
draw(L = Label("$\theta$", align = NW), arc(C, C + 3 * v, C + 3 * xp));

// gravity
draw(L = Label("$g$", align = E), C -- C - 5Y, Arrow);
draw(C -- C + 5Y, dashed);

// origin
label("$O$", C, E);
\end{asy}
\end{center}

So this problem is easier than the last problem! There is no normal
force to worry about and the components of gravity are fairly easy.
\[ \vec{F} = -mg \hat{y}' = -mg (\sin \phi \hat{x} + \cos \phi \hat{y}).
\]
This gives us the nice equations
\[ \left.\begin{aligned}
F_x &= -mg \sin \phi = m \ddot{x} \\
\implies x &= -\frac{g}{2} \sin \phi t^2 + v_0 \cos \theta t \\
F_y &= -mg \cos \phi = m \ddot{y} \\
\implies y &= -\frac{g}{2} \cos \phi t^2 + v_0 \sin \theta t. \\
\end{aligned} \right \rbrace
\implies
\boxed{\vec{r} = \left(-\frac{g}{2} \sin \phi t^2 + v_0 \cos \theta t,
-\frac{g}{2} \cos \phi t^2 + v_0 \sin \theta t\right)}. \quad
\textrm{(a)} \]
From this, we see that $y = 0$ at $t = \frac{2 v_0 \sin \theta}{g \cos
\phi}$ and plugging into the $x$ equation, we get
\[ \begin{aligned}
x &= -\frac{g}{2} \sin \phi \left(\frac{4 v_0^2 \sin^2 \theta}{g^2
\cos^2 \phi}\right) + v_0 \cos \theta \left(\frac{2 v_0 \sin \theta}{g
\cos \phi}\right) \\
&= -\frac{2 v_0^2 \sin^2 \theta \sin \phi}{g \cos^2 \phi} + \frac{2
v_0^2 \sin \theta \cos \theta \cos \phi}{g \cos^2 \phi} \\
&= \boxed{\frac{2 v_0^2 \sin \theta \cos(\theta + \phi)}{g \cos^2
\phi}}. \quad \textrm{(b)}
\end{aligned} \]
To maximize $x$, we must maximize the numerator because it is fixed.
Thus we differentiate with respect to $\theta$ and get
\[ \frac{dx}{d \theta} = 2 v_0^2 (\cos \theta \cos(\theta + \phi) - \sin
\theta \sin(\theta + \phi)) = 2 v_0^2 \cos (2 \theta + \phi), \]
which vanishes when $2 \theta + \phi = \frac{\pi}{2}$. Now we
can plug into
\[ \begin{aligned}
x &= \frac{2 v_0^2 \sin \theta \cos(\theta + \phi)}{g \cos^2 \phi} \\
&= \frac{v_0^2 (\sin -\phi + \sin(2 \theta + \phi))}{g \cos^2 \phi} \\
&= \frac{v_0^2 (1 - \sin \phi)}{g (1 - \sin^2 \phi)} \\
\Aboxed{ &= \frac{v_0^2}{g (1 + \sin \phi)}}, \quad \textrm{(c)} \\
\end{aligned} \]
as desired.
\end{proof}

\begin{prb}[A New Take on Projectiles]
You launch a projectile at an angle $\theta$ with initial velocity
$v_0$. Neglecting air resistance, determine
\begin{enumerate}[(a)]
\item The ball's position $\vec{r}(t)$ with $x$ being the horizontal
position and $y$ being the vertical position.
\item The maximum $\theta$ such that $|\vec{r}|$ is non-decreasing.
\end{enumerate}
\end{prb}

\begin{proof}[Solution]
This is the classical projectile motion problem, which we solved in
problem 1 with slightly different parameters.
\[ \vec{r} = \left(v_0 \cos \theta t, -\frac{g}{2} t^2 + v_0 \sin \theta
t \right). \quad \textrm{(a)} \]
A non-decreasing norm is equivalent to the square of the norm
non-decreasing so we want
\[ v_0^2 \cos^2 \theta t^2 + \frac{g^2}{4} t^4 - v_0 g \sin \theta t^3 +
v_0^2 \sin^2 \theta t^2  = v_0^2 t^2 + \frac{g^2}{4} t^4 - v_0 g \sin
\theta t^3 \]
to be non-decreasing, so we differentiate to get
\[ \begin{aligned}
2v_0^2 t + g^2 t^3 - 3 v_0 g \sin \theta t^2 &\geq 0 \\
2v_0^2 t + g^2 t^3 &\geq 3 v_0 g \sin \theta t^2 \\
\frac{2 v_0^2 + g^2 t^2}{3 v_0 g t} &\geq \sin \theta \\
\implies \min \frac{2 v_0^2 + g^2 t^2}{3 v_0 g t} &\geq \sin \theta. \\
\end{aligned} \]
We now differentiate the LHS to compute the minimum, which occurs at
\[ \begin{aligned}
(3v_0 gt) (2 g^2 t) - (2 v_0^2 + g^2 t^2)(3v_0 g) &= 0 \\
6 v_0 g^3 t^2 - 6v_0^3 g - 3 v_0 g^3t^2 &= 0 \\
v_0 g^3 t^2 &= 2 v_0^3 g \\
t^2 &= \frac{2 v_0^2}{g^2}, \\
\end{aligned} \]
which allows us to plug back in.
\[ \begin{aligned}
\sin \theta &\leq \frac{2v_0^2 + 2v_0^2}{3 v_0 g \sqrt{2} \dfrac{v_0}{g}} \\
\sin \theta &\leq \frac{4}{3 \sqrt{2}} = \frac{2 \sqrt{2}}{3} \\
\implies \Aboxed{\theta &= \arcsin \frac{2 \sqrt{2}}{3}} \quad
\textrm{(b)} \\
\end{aligned} \]
\end{proof}

\begin{prb}[Cylindrical Space Fun]
Consider a mass $m$ being twirled in a circle at the end of a massless
string of length $R$ with constant angular velocity $\omega$ in an
inertial reference frame and a gravity free environment. Use a
coordinate system such that the fixed end of the string is at the origin
and the mass is orbiting on the plane $z = 0$.
\begin{enumerate}[(a)]
\item Write down Newton's second law in cylindrical coordinates (polar
coordinates in this case) and determine the string tension.
\item Supose $\phi = 0$ at $t = 0$ and $\dot{\phi} \geq 0$. At $t =
\frac{2 \pi}{3 \omega}$, the string breaks. Determine the mass's
trajectory after this time.
\end{enumerate}
\end{prb}

\begin{proof}[Solution]
Here is a diagram.
\begin{center}
\begin{asy}
size(7cm);

pair O  = (0, 0);
pair X  = (1, 0);
pair Y  = (0, 1);
path c  = unitcircle;
path dc = scale(0.25) * c;
real ps = 0.4;
pair b  = point(dc, 0);
pair e  = point(dc, ps);
pair p  = point(c, ps);

draw(unitcircle);
draw(L = Label("$\phi = \omega t$", align = unit(3E + N)), arc(O, b, e),
Arrow);

draw(L = Label("$R$", align = S), O -- X, dashed);
draw(O -- unit(e));

// Ft
pair ft = 0.6 * unit(O - e);
draw(p -- p + 0.6 * unit(O - e), black + 1.25);
draw(L = Label("$\vec{F}_T$", align = 1.7N), p -- p + ft, Arrow);

// v
draw(L = Label("$\vec{v}$", align = E), p -- p + rotate(-90) * ft,
Arrow);

// m
dot(p);
label("$m$", p, NE);

// other things
label("$O$", O, W);
label("$\phi = 0$", X, E);
\end{asy}
\end{center}
The diagram is representative of a system constrained by $s = R, \phi =
\omega t$, so we have something like
\[ \boxed{\vec{F}_T = \vec{F} = m \ddot{\vec{r}} = m \left( \left(
\ddot{s} - s \dot{\phi}^2\right) \hat{s} + \left(s \ddot{\phi} + 2
\dot{s} \dot{\phi} \right) \hat{\phi}\right) = -mR \omega^2 \hat{s}}.
\quad \textrm{(a)} \]
Our velocity is just
\[ \vec{v} = \dot{s} \hat{s} + s \dot{\phi} \hat{\phi} = R \omega
\hat{\phi} \]
which implies that our velocity when the string breaks and there is no
force is
\[ \vec{r} = R \hat{s} + R \omega \hat{\phi} t, \]
where $\hat{s}, \hat{\phi}$ are representatives of time $t = \frac{2
\pi}{3 \omega}$. Converting back into cartesian coordinates, this is
simply
\[ \begin{aligned}
\vec{r} &= R\left(\cos \frac{2 \pi}{3 \omega}, \sin \frac{2 \pi}{3
\omega}\right) + R \omega t \left(\sin \frac{2 \pi}{3 \omega}, -\cos
\frac{2 \pi}{3 \omega}\right) \\
\Aboxed{&= R \left(\cos \frac{2 \pi}{3 \omega} + \omega \sin \frac{2
\pi}{3 \omega} t, \sin \frac{2 \pi}{3 \omega} - \omega \cos \frac{2
\pi}{3 \omega} t\right)} \quad \textrm{(b)} \\
\end{aligned} \]
\end{proof}

\begin{prb}[More Cylinders]
Again, we have a point mass $m$ constrained to move around a single
point $O$ at a distance $R$, but this time, there may be some sort of
gravity acting on the mass.
\begin{enumerate}[(a)]
\item Determine the general motion of the mass with Newton's second law
in cylindrical coordinates.
\item Suppose gravity is the peculiar function $\vec{g} = -mKt\hat{z}$
with $K$ being a constant. Determine $z(t)$ with $z(0) = 0$.
\item We now live in a gravity free region of space and experience a
drag force equivalent to $\vec{F}_d = -K \vec{v}$ with $\vec{v}$ being
the velocity of the mass. We start at $(s, \phi, z) = (R, 0, 0)$ with
initial velocity $\vec{v}(0) = R \omega_0 \hat{\phi}$ with constant
$\omega_0$. Determine $\vec{r}(t) \quad \forall t \geq 0$.
\item Using (c), determine if the mass ever stops. If it does, compute
the stopping time $t$ and the total distance travelled before stopping.
\end{enumerate}
\end{prb}

\begin{proof}[Solution]
I believe we proceed assuming $\ddot{\phi} = 0$. If not, then we have a
system of differential equations for $\phi$. Again, we have by
Newton's second law,
\[ \begin{aligned}
\vec{F} &= F_N \hat{s} - g \hat{z} \\
\vec{F} &= m \ddot{\vec{r}} \\
&= m \left(\left( \ddot{s} - s \dot{\phi}^2 \right) \hat{s} + \left(s
\ddot{\phi} + 2 \dot{s} \dot{\phi}\right) \hat{\phi} + \ddot{z}
\hat{z}\right) \\
&= m\left(-R \dot{\phi}^2 \hat{s} + \ddot{z} \hat{z}\right) \\
\end{aligned} \implies \boxed{\left \lbrace \begin{aligned}
s &= R \\
\phi &= \dot{\phi} t + \phi_0 \\
z &= -\frac{g}{2} t^2 + v_{0_z} t + z_0 \\
\end{aligned} \right.}. \quad \textrm{(a)} \]

For the weird gravitational field, we have the differential equation
\[ \ddot{z} = -Kt. \]
This has a standard solution of $\boxed{z = -\frac{K}{6} t^3 + v_{0_z}
t} \quad \textrm{(b)}$ via integration.

With a drag force $R \ddot{\phi} \hat{\phi} = \vec{F}_d = -K \vec{v} =
-K (\dot{s} \hat{s} + s \dot{\phi} \hat{\phi} + \dot{z} \hat{z}) = -KR
\dot{\phi}\hat{\phi}$, we have $\ddot{\phi} = \frac{d \dot{\phi}}{dt} =
-K \dot{\phi} \implies \dot{\phi} = \omega_0 e^{-K t} \implies \phi =
\frac{\omega_0}{K} \left(1 - e^{-Kt}\right) \implies \boxed{(s, \phi, z)
= \left(R, \frac{\omega_0}{K}\left(1 - e^{-Kt}\right), 0\right)} \quad
\textrm{(c)}$.

From the above, it is then easy to see that $\lim_{t \to \infty} \phi =
\frac{\omega_0}{K}$, so the puck does stop (eventually) with parameters
$\boxed{d(t) \xrightarrow[t \to \infty]{} \frac{R \omega_0}{K}} \quad
\textrm{(d)}$.
\end{proof}

\end{document}
