\documentclass{article}

\usepackage[margin=1in]{geometry}
\usepackage{amsmath}
\usepackage{amsthm}
\usepackage{amsfonts}
\usepackage{amssymb}
\usepackage{asymptote}
\usepackage{fancyhdr}
\usepackage{mathtools}
\usepackage{parskip}

% setup the header
\pagestyle{fancy}
\lhead{Will Song}
\chead{Phys 325}
\rhead{HW1}

% setup definition/theorem etc
\newtheoremstyle{norm}
{3pt}
{3pt}
{}
{}
{\bf}
{:}
{.5em}
{}

\theoremstyle{norm}
\newtheorem{thm}{Theorem}[section]
\newtheorem{lem}[thm]{Lemma}
\newtheorem{df}[thm]{Definition}
\newtheorem{rem}[thm]{Remark}
\newtheorem{st}{Step}
\newtheorem{prop}[thm]{Proposition}
\newtheorem{cor}[thm]{Corollary}
\newtheorem{conj}[thm]{Conjecture}
\newtheorem{clm}[thm]{Claim}
\newtheorem{exr}[thm]{Exercise}
\newtheorem{ex}[thm]{Example}
\newtheorem{prb}[thm]{Problem}
\newtheorem{note}[thm]{Note}

% just useful shorthand
\renewcommand{\st}{\,\operatorname{s.t.}\,}
\let\hom\relax
\DeclareMathOperator{\hom}{Hom}
\DeclareMathOperator{\Tr}{Tr}
\DeclareMathOperator{\sgn}{sgn}
\DeclareMathOperator{\adj}{adj}
\DeclareMathOperator{\proj}{proj}
\newcommand{\pvec}[1]{\vec{#1}\mkern2mu\vphantom{#1}}

% for easier math
\everymath{\displaystyle}

\begin{document}

\section{Homework 1}

\begin{prb}[Projectile Motion]
Consider a inclined plane angled at an angle of $\theta$ with respect to
the $xy$ plane. Call one corner touching the ground the origin and the
edge lying against the ground the $x$ axis. The $y$ axis runs along the
inclined edge of the inclined plane and the $z$ axis is normal to the
plane. A frictionless puck at rest is kicked so that it slides along the
plane with initial velocity $\vec{v}_0 = v_{0_x} \hat{x} + v_{0_y}
\hat{y}$ and normal gravity $g$. Compute how far the puck travles once
it returns to floor level.
\end{prb}

\begin{proof}[Solution]
We begin by drawing a very nice diagram.
\begin{center}
\begin{asy}
import three;
import graph3;
import solids;

settings.render = 0;
settings.prc = false;
settings.multisample = 1;

size(10cm);

currentprojection = perspective(30, 25, 8);

real z = 7;
real x = 25;
triple o = x * X + 24 Y;
path3 pl = z * Z -- x * X + z * Z -- o -- 24Y -- cycle;

// draw axes
draw(L = Label("$x'$", position = Relative(1.1), align = SW), O -- x*X,
red + 0.75, Arrow3);
draw(L = Label("$y'$", position = Relative(1.1), align = SE), O -- 28Y,
red + 0.75, Arrow3);
draw(L = Label("$z'$", position = Relative(1.1), align =  N), O --
(z+1)*Z, red + 0.75, Arrow3);

// draw plane
draw(surface(O -- x*X -- x*X + 24Y -- 24Y -- cycle), emissive(gray +
opacity(0.4)));
draw(surface(pl), emissive(gray + opacity(0.5)));

// compute new unit vectors
triple xp = unit(24Y - o);
triple yp = unit(x*X + z*Z - o);
triple zp = cross(xp, yp);

// draw new coordinates
draw(L = Label("$x$", position = Relative(1.1), align = S), o -- o + 6
xp, blue + 0.75, Arrow3);
draw(L = Label("$y$", position = Relative(1.1), align = W), o -- o + 6
yp, blue + 0.75, Arrow3);
draw(L = Label("$z$", position = Relative(1.1), align = E), o -- o + 6
zp, blue + 0.75, Arrow3);
dot(L = Label("$O$", align = SW), o);

// draw initial vector
path3 p = o -- o + 6 xp + 8 yp;
triple dir = unit(6 xp + 8 yp);
draw(L = Label("$\vec{v}_0 = v_{0_x} \hat{x} + v_{0_y} \hat{y}$",
position = Relative(1.6), align = unit(NW + 2N)), p, Arrow3);

// mark angle to vector
triple pp = o + 2 dir;
triple xx = o + 2 xp;
triple yy = o - 2 Y;
triple yz = o + 2 yp;

draw(arc(o, pp, xx));
draw(L = Label("$\theta$", align = SW), arc(o, yz, yy));

label(Label("$m$", align = SE), o);

// draw gravity
draw(L = Label("$g$", position = Relative(0.5), align = E), o -- o -
5Z, Arrow3);
\end{asy}
\end{center}

Again we refer back to the magical law $\vec{F} = m \ddot{\vec{r}}$ and
realize this is fairly simple as the only force we experience here is
gravity and the normal force. In other words,
\[ \begin{aligned}
\vec{F} = -mg \hat{z}' + \vec{F}_N &= -mg(\proj_{\hat{x}} \hat{z}' +
\proj_{\hat{y}} \hat{z}' + \proj_{\hat{z}} \hat{z}') + \vec{F}_N \\
&= -mg\left( (\hat{y} \cdot \hat{g}) \hat{y} + k \hat{z} \right) +
\vec{F}_N \\
&= -mg \left( \sin \theta \hat{y} + k \hat{z} \right) + \vec{F}_N.
\end{aligned} \]

Luckily $\vec{F}_N$ is constrained along the $z$ axis such that $z = 0$
so we can disregard that and just look at the $y$ coordinate. There is
no acceleration in the $x$ direction so we just have $x = v_{0_x} t$.
However, the $y$ coordinate experiences constant acceleration $-mg \sin
\theta$ so we have $y = - \frac{g}{2} \sin \theta t^2 + v_{0_y} t$ which
reaches its maximum at $t = \frac{2 v_{0_y}}{g \sin \theta}$ So the
distance from the $x$ coordinate is just
\[ \boxed{\frac{2 v_{0_x} v_{0_y}}{g \sin \theta}}, \]
which is what we would expect from normal projectile motion.
\end{proof}

\end{document}
