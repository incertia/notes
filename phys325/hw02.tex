\documentclass{article}

\usepackage{incertia}

% setup the header
\pagestyle{fancy}
\lhead{Will Song}
\chead{Phys 325}
\rhead{HW2}

\begin{document}

\section{Homework 2}

\begin{prb}[Mixed Dependencies]
A particle of mass $m$ moves under the influence of $\vec{F} = -ktv
\hat{v}$ where $k$ is a positive constant. At time $t = 0$ the particle
passes through the origin with velocity $v_0 \hat{x}$, where $v_0$ is a
positive constant.
\begin{enumerate}[(a)]
\item Calculate the particle's velocity $\vec{v}(t)$ for $t \geq 0$.
\item How much time does it take for the particle to stop?
\item What is the maximum $x$ position reached by the particle?
\end{enumerate}
\end{prb}

\begin{proof}[Solution]
We have $-kt\vec{v} = -ktv \hat{v} = \vec{F} = m\vec{a} = m \frac{d
\vec{v}}{dt}$, which generalizes to $-ktv_i = m \frac{dv_i}{dt}$ due to
Cartesian coordinates. This simplifies to
\[ -\frac{k}{m}tdt = \frac{dv_i}{v_i}, \]
which has the convenient solution
\[ \begin{aligned}
&& -\frac{k}{2m}t^2 &= \ln v_i + C \\
&\implies& Ae^{-\frac{k}{2m}t^2} &= v_i \\
&\implies& v_i &= v_{i_0} \exp\left(-\frac{k}{2m}t^2\right). \\
\end{aligned} \]
Combine this with $\vec{v}_0 = (v_0, 0, 0)$ to get
\[ \boxed{\vec{v} = v_0 \exp \left(-\frac{k}{2m}t^2\right) \hat{x}}.
\quad \textrm{(a)} \]
Due to exponential greatness, it takes infinite time for the particle to
stop, or $\boxed{v \xrightarrow[t \to \infty]{} 0} \quad \textrm{(b)}$.

The $x$ component of $\vec{v}$ never changes sign (exponential
greatness) so we just compute the integral from $0$ to $\infty$.
\[ \begin{aligned}
x &= v_0 \int_0^\infty e^{-\frac{k}{2m}t^2} dt \\
&= \boxed{\sqrt{\frac{m \pi}{2k}}} \quad \textrm{(c)}
\end{aligned} \]
\end{proof}

\begin{prb}[Position Dependent?]
A particle of mass $m$ is constrained to move along the positive $x$
axis under the influence of $\vec{F} = -\frac{K}{x^2}\hat{x}, K > 0$. At
time $t = 0$, the particle is released from rest at position $x_0 > 0$.
Compute the first time where $x = 0$.
\end{prb}

\begin{proof}[Solution]
We begin by writing
\[ -\frac{K}{x^2} = F = ma = m \frac{dv}{dt} = m \frac{dv}{dt}
\frac{dx}{dx} = m v \frac{dv}{dx}. \]
This simplies into
\[ \begin{aligned}
&& -\frac{Kdx}{x^2} &= mvdv \\
&\implies& \frac{K}{x} &= \frac{1}{2}mv^2 + C \\
&\implies& \frac{K}{x} &= \frac{1}{2}mv^2 + \frac{K}{x_0} \\
&& v^2 &= \frac{2K}{m}\left(\frac{1}{x} - \frac{1}{x_0}\right) \\
&\implies& \frac{dx}{dt} &= -\sqrt{\frac{2K}{m}\left(\frac{1}{x} -
\frac{1}{x_0}\right)} \\
\end{aligned} \]
We now make the substitution $x = x_0 \sin^2 \alpha$ because we only
care about $x \geq 0$ and we don't really care about whether or not $x <
-x_0$. This gives us
\[ \begin{aligned}
\frac{dx}{dt} &= -\sqrt{\frac{2K}{m x_0} \left( \frac{1}{\sin^2 \alpha}
- 1 \right)} \\
&= -\sqrt{\frac{2K}{mx_0}} \cot \alpha \\
\frac{d(x_0 \sin^2 \alpha)}{- \sqrt{\frac{2K}{mx_0}} \cot \alpha} &= dt
\\
t &= \int_{\dfrac{\pi}{2}}^0 \frac{2 x_0 \sin \alpha \cos \alpha d
\alpha}{-\sqrt{\dfrac{2K}{mx_0}} \cot \alpha} \\
&= 2 \int_{\dfrac{\pi}{2}}^0 \frac{x_0 \sin^2 \alpha d \alpha}{-\sqrt{
\dfrac{2K}{mx_0}}} \\
&= \int_0^{\dfrac{\pi}{2}} \sqrt{\frac{2mx_0^3}{K}} \sin^2 \alpha d
\alpha \\
&= \sqrt{\frac{2mx_0^3}{K}} \int_0^{\dfrac{\pi}{2}} \frac{1 - \cos(2
\alpha)}{2} d \alpha \\
&= \boxed{\sqrt{\frac{mx_0^3}{8K}} \pi}.
\end{aligned} \]
\end{proof}

\begin{prb}[More Drag]
Suppose a block with mass $m$ slides on a frictionless surface with
speed $v_0$ in the $+x$ direction at $t = 0$. The block experiences a
drag force of form
\[ \vec{F}(v) = -k v^\beta \hat{v}, \]
where $k, \beta$ are positive constants.

Where does the block stop and how long does it take to stop for the
following cases?
\begin{enumerate}[(a)]
\item $\beta = 0$.
\item $\beta = \frac{3}{2}$.
\item $\beta = \frac{1}{2}$. Also change the experiment so that the
block is now falling downwards so it also experiences gravity.
Experimenters determine that the block's terminal velocity
$v_{\textrm{TER}} = \SI{50}{m/s}$. Approximate $g = \SI{10}{m/s^2}$ and
compute the times when the block reaches $50\%, 90\%, 99\%$, and
$99.9\%$ of its final speed $v_{\textrm{TER}}$.
\end{enumerate}
\end{prb}

\begin{proof}[Solution]
$ $
\begin{enumerate}[(a)]
\item
We start out by plugging. If $\beta = 0$, then we have constant drag
force
\[ \vec{F} = -k\hat{v}, \]
which simplifies to
\[ dv = -\frac{k}{m} dt, \]
or
\[ v = -\frac{k}{m} t + v_0, \]
which has zero $t = \frac{mv_0}{k}$.  We integrate again to find
\[ x = -\frac{k}{m}t^2 + v_0 t, \]
so the block stops at
\[ \boxed{t = \frac{v_0 m}{k} \quad x = \frac{mv_0^2}{2k}}. \]
\item
Again we start by substituting for $\beta$.
\[ \begin{aligned}
m \frac{dv}{dt} &= -k v^{\frac{3}{2}} \\
v^{-\frac{3}{2}} dv &= -\frac{k}{m} dt, \\
\end{aligned} \]
which integrates to
\[ -2v^{-\frac{1}{2}} + c = -\frac{k}{m}t, \]
which, after solving for $c$ by plugging in $t = 0$, becomes
\[ \frac{1}{\sqrt{v}} - \frac{1}{\sqrt{v_0}} = \frac{k}{2m} t, \]
which rearranges nicely to form
\[ \frac{1}{\sqrt{v}} = \frac{k}{2m} t + \frac{1}{\sqrt{v_0}}. \]
Inverting and squaring, we arrive at
\[ \begin{aligned}
\frac{dx}{dt} = v &= \frac{1}{\dfrac{k^2 t^2}{4m^2} +
\dfrac{kt}{m\sqrt{v_0}} + \dfrac{1}{v_0}} \\
&= \frac{v_0}{\dfrac{k^2t^2v_0}{4m^2} + \dfrac{kt\sqrt{v_0}}{m} + 1} \\
dx &= \frac{4m^2 v_0 dt}{k^2 t^2 v_0 + 4mkt \sqrt{v_0} + 4m^2} \\
&= \frac{4m^2 v_0 dt}{\left(kt \sqrt{v_0} + 2m\right)^2}, \\
\end{aligned} \]
but fear not, because we can use the very nice substitution
\[ u = kt\sqrt{v_0} + 2m \quad du = k \sqrt{v_0} dx \]
to get
\[ \begin{aligned}
dx &= \frac{4m^2 \sqrt{v_0} du}{k u^2} \\
x &= -\frac{4m^2 \sqrt{v_0}}{ku} \\
&= -\frac{4m^2\sqrt{v_0}}{k^2 t\sqrt{v_0} + 2km} +
\frac{2m\sqrt{v_0}}{k}. \\
\end{aligned} \]
An equivalent form would be
\[ x = \frac{2m}{k} \left(\sqrt{v_0} - \frac{1}{\dfrac{1}{\sqrt{v_0}} +
\dfrac{kt}{2m}} \right). \]
By observation of the velocity equation, we notice that $v
\xrightarrow{t \to \infty} 0$, so we plug into the position equation to
get
\[ \boxed{t \to \infty \quad x = \frac{2m\sqrt{v_0}}{k}}. \]
\item
Again we start by substituting for $\beta$, but this time we also add in
gravity.
\[ F = -k\sqrt{v} + mg. \]
We reach terminal velocity when there is no force, or $mg = k \sqrt{v}$,
or
\[ v_{\textrm{TER}} = \left(\frac{mg}{k}\right)^2. \]
We will store this equation in our memory for now. Next, we determine a
nice form for velocity by using the force equation again.
\[ \frac{dv}{dt} = g - \frac{k}{m}\sqrt{v} = g \left(1 -
\frac{k}{mg}\sqrt{v}\right) = g\left(1 - \sqrt{\frac{v}{v_T}}\right). \]
This rearranges to
\[ \frac{dv}{1 - \sqrt{\dfrac{v}{v_T}}} = g dt, \]
which integrates to
\[ gt = \left. -2v_T \left(\sqrt{\frac{v}{v_T}} + \ln
\left(\sqrt{\frac{v}{v_T}} - 1\right)\right) \right \vert_0^v = -2v_T
\left(\sqrt{\frac{v}{v_t}} + \ln \left(1 - \sqrt{\frac{v}{v_T}}\right)
\right). \]
By setting $\frac{v}{v_T} \in \lbrace 0.5, 0.9, 0.99, 0.999 \rbrace$, we
get $\boxed{t = \SI{5.21}{s}, \SI{20.21}{s}, \SI{43.01}{s},
\SI{66.01}{s}}$.
\end{enumerate}
\end{proof}

\begin{prb}[Velocity Dependent, but not Drag]
A particle with charge $q$ moves through a constant magnetic field with
$\vec{B} = B \hat{z}, B > 0$, so the particle experiences Lorentz force
$\vec{F} = q\vec{v} \times \vec{B}$. Assume the effects of gravity are
trivial. Let the particle's intial velocity at $t = 0$ be $\vec{v}_0 =
(v_{0_x}, v_{0_y}, v_{0_z})$ and position be $\vec{r}_0 = (x_0, y_0,
0)$.

\begin{enumerate}[(a)]
\item
Warmup question. Show that expressions of the form $A \cos \omega
t + B \sin \omega t$ are interchangeable with expressions of the form $C
\sin(\omega t + \phi)$.
\item
Use Newton's Laws to obtain differential equations describing the
particle's velocity in $x, y, z$ componenets.
\item
Compute $\vec{v}(t)$.
\item
Compute $\vec{r}(t)$.
\item
The trajectory obtained is called a \textbf{helix}. Sketch it and
calculate its two dimension, diameter and pitch (rise per full turn).
\end{enumerate}
\end{prb}

\begin{proof}[Solution]
To switch between the two, we need magnitutdes of both to be equal, but
that's easy because we can expand to get $A = C\sin\phi, B = C\cos\phi$.
This yields the obvious transformations
\[\boxed{C = \sqrt{A^2 + B^2} \quad \phi = \arctan \frac{A}{B}.} \quad
\textrm{(a)} \]
Next, we compute $\vec{F} = q\vec{v} \times \vec{B} = q(v_x, v_y, v_z)
\times (0, 0, B) = q(Bv_y, -Bv_x, 0)$ and we obtain partial differential
equations
\[ \boxed{\left\lbrace \begin{aligned}
\frac{\partial v_x}{\partial t} &= \frac{qB}{m} v_y \\
\frac{\partial v_y}{\partial t} &= -\frac{qB}{m} v_x \\
\frac{\partial v_z}{\partial t} &= 0. \\
\end{aligned}\right.} \quad \textrm{(b)} \]
The coupling of the partial derivatives motivates us to apply
$\frac{\partial}{\partial t}$ to each equation, which gives us a better
set of partial differential equations.
\[ \left\lbrace \begin{aligned}
\frac{\partial^2 v_x}{\partial t^2} &= -\left(\frac{qB}{m}\right)^2 v_x
\\
\frac{\partial^2 v_y}{\partial t^2} &= -\left(\frac{qB}{m}\right)^2 v_y
\\
\frac{\partial^2 v_z}{\partial t^2} &= 0, \\
\end{aligned} \right. \]
which have the very natural solution
\[ \boxed{\left\lbrace \begin{aligned}
v_x       &= v \sin(\omega t + \phi) \\
v_y       &= v \cos(\omega t + \phi) \\
v_z       &= v_{0_z} \\
\end{aligned} \quad \begin{aligned}
v         &= \sqrt{v_x^2 + v_y^2} \\
\omega    &= \frac{qB}{m} \\
\tan \phi &= \frac{v_{0_x}}{v_{0_y}}. \\
\end{aligned} \right.} \quad \textrm{(c)} \]
\textbf{Note}: The best way to see the $\tan \phi$ condition is to set
$t = 0$.

To find position, we just integrate normally (for example, $r_x = x_0 +
\int_0^t v \sin(\omega s + \phi) ds$) to arrive at
\[ \boxed{\left\lbrace \begin{aligned}
r_x &= x_0 - \frac{v}{\omega}\cos(\omega t + \phi) +
\frac{v_{0_y}}{\omega} \\
r_y &= y_0 + \frac{v}{\omega}\sin(\omega t + \phi) -
\frac{v_{0_x}}{\omega} \\
r_z &=  v_{0_z} t. \\
\end{aligned} \right.} \quad \textrm{(d)} \]

Here is a picture of a helix starting on the $x$ axis.
\begin{center}
\begin{asy}
import three;
import graph3;
import solids;

settings.render = 0;
settings.prc = false;
settings.multisample = 1;

size(5cm);

currentprojection = perspective(10, 10, 10);

// helix functions
real hx(real t) { return 4 * cos(2pi * t); }
real hy(real t) { return 4 * sin(2pi * t); }
real hz(real t) { return t; }

// draw axes
draw(L = Label("$x$", position = Relative(1.1), align = SW), O -- 5X,
black + 0.75, Arrow3);
draw(L = Label("$y$", position = Relative(1.1), align = SE), O -- 5Y,
black + 0.75, Arrow3);
draw(L = Label("$z$", position = Relative(1.1), align =  N), O -- 6Z,
black + 0.75, Arrow3);

// draw helix
draw(graph(hx, hy, hz, 0, 4, operator ..), Arrow3);
\end{asy}
\end{center}
Our particular helix has parameters $d = 2r = 2\frac{v}{\omega} =
\frac{2m}{qB}\sqrt{v_{0_x}^2 + v_{0_y}^2}$ and pitch $\Delta z = v_{0_z}
T = v_{0_z} \frac{2\pi}{\omega} = \frac{2 \pi m}{qB} v_{0_z}$.
\end{proof}

\end{document}
