\section{Day 2}

\subsection{Cylindrical Coordinates}
\begin{center}
\begin{asy}
import three;
import graph3;

settings.render = 0;
settings.prc = false;

size(8cm);

currentprojection = perspective(5, 1, 3.5);

real z = 4;
triple p  = (0, 0, z);
triple q  = (1, sqrt(3), z);
triple x  = (1, sqrt(3), 0);
triple ap = point(O -- x, 0.3);
triple aq = rotate(-60, Z) * ap;
//path c = 2X .. 2Y .. -2X .. -2Y .. cycle;

draw(L = Label("$x$", position = Relative(1.1), align = SW), O -- 5 X,
red + 0.75, Arrow3);
draw(L = Label("$y$", position = Relative(1.1), align = SE), O -- 5 Y,
blue + 0.75, Arrow3);
draw(L = Label("$z$", position = Relative(1.1), align =  N), O -- 6 Z,
green + 0.75, Arrow3);

draw(L = Label("$\vec{s}$", position = Relative(0.5), align = N), p --
q, Arrow3);
draw(L = Label("$\vec{r}$", position = Relative(0.5), align = E), O --
q, Arrow3);

draw(O -- x);
draw(L = Label("$z$", position = Relative(0.5), align = E), q -- x, dashed);
draw(L = Label("$\phi$", position = Relative(0.5), align = (0.0005,
-0.0005)), arc(O, ap, aq));

draw(2X .. 2Y .. -2X .. -2Y .. cycle, dashed);
draw(2X + z * Z .. 2Y + z * Z .. -2X + z * Z .. -2Y + z * Z .. cycle,
dashed);

//draw(rotate(180, Z) * q -- rotate(180, Z) * x, dashed);

dot(q);

label("$P = (s, \phi, z)$", q, SE);
\end{asy}
\end{center}

We have the coordinates $(s, \phi, z)$ where $s$ is the distance from
the vertical axis, $\phi$ is the rotational angle with respect to the
plane perpendicular to the vertical axis, and $z$ is the position on the
vertical axis. This gives us a nice definition for the position vector.
\[ \vec{r} = s \hat{s} + z \hat{z} \]

\begin{note}
The unit vectors $\hat{s}, \hat{\phi}$ are position dependent! More
specifically, they depend on $\phi$.

We have
\[ \frac{\partial \hat{s}}{\partial \phi} = \hat{\phi} \quad
\frac{\partial \hat{\phi}}{\partial \phi} = -\hat{s}. \]
\end{note}

\begin{cor}
We then have the line element
\[ d \vec{r} = d \vec{l} = \left( \frac{\partial}{\partial s} ds +
\frac{\partial}{\partial \phi} d \phi + \frac{\partial}{\partial z} dz
\right) \vec{r} = ds \hat{s} + s d \phi \hat{\phi} + dz \hat{z}, \]
% \[ d \vec{r} = d \vec{l} = d(s \hat{s} + z \hat{z}), \]
velocity
\[ \vec{v} = \frac{d \vec{r}}{dt} = \frac{ds}{dt} \hat{s} + s \frac{d
\phi}{dt} \hat{\phi} + \frac{dz}{dt} \hat{z} = \dot{s} \hat{s} + s
\dot{\phi} \hat{\phi} + \dot{z} \hat{z}, \]
and acceleration
\[ \begin{aligned}
\vec{a} = \frac{d \vec{v}}{dt} &= \frac{d}{dt} \left( \dot{s} \hat{s}
+ s \dot{\phi} \hat{\phi} + \dot{z} \hat{z} \right) \\
&= \ddot{s} \hat{s} + \dot{s} \frac{d \hat{s}}{dt}  + \dot{s} \dot{\phi}
\hat{\phi} + s \ddot{\phi} \hat{\phi} + s \dot{\phi} \frac{d
\hat{\phi}}{dt} + \ddot{z} \hat{z} + \dot{z} \frac{d \hat{z}}{dt} \quad
\textrm{(product rule rocks)} \\
&= \ddot{s} \hat{s} + \dot{s} \frac{d \hat{s}}{d \phi}\frac{d \phi}{dt}
+ \dot{s} \dot{\phi} \hat{\phi} + s \ddot{\phi} \hat{\phi} + s
\dot{\phi} \frac{d \hat{\phi}}{d \phi} \frac{d \phi}{dt} + \ddot{z}
\hat{z} \\
&= \ddot{s} \hat{s} + \dot{s} \hat{\phi} \dot{\phi} + \dot{s} \dot{\phi}
\hat{\phi} + s \ddot{\phi} \hat{\phi} - s \dot{\phi} \hat{s} \dot{\phi}
+ \ddot{z} \hat{z} \\
&= \left( \ddot{s} - s \dot{\phi}^2 \right) \hat{s} + \left( s
\ddot{\phi} + 2 \dot{s} \dot{\phi} \right) \hat{\phi} + \ddot{z} \hat{z}
\end{aligned} \]
\end{cor}

\begin{rem}
Notice here that $d \hat{s}$ refers to the total differential of
$\hat{s}$, as well as $d \hat{\phi}$ and $d \hat{z}$. Where total
differential of a function $f : \mathbb{R} \times \mathbb{R} \times
\cdots \times \mathbb{R} \to \mathbb{R}$ is defined as
\[ df(x_1, x_2, \dots, x_n) = \left( \sum_{i = 1}^n
\frac{\partial}{\partial x_i} d x_i \right) f. \]
\end{rem}

\begin{ex}
We have a trajectory
\[ \begin{aligned}
s(t) &= R, \\
\phi(t) &= \omega t, \\
z(t) &= \alpha t. \\
\end{aligned} \]
Compute $|\vec{v}(t)|$ and $\vec{a}$.
\end{ex}

\begin{proof}[Solution]
We simply plug.
\[ \begin{aligned}
\vec{v} &= \dot{s} \hat{s} + s \dot{\phi} \hat{\phi} + \dot{z} \hat{z}
\\
&= 0 \hat{s} + R \omega \hat{\phi} + \alpha \hat{z} \\
&\Downarrow \\
\Aboxed{|\vec{v}| &= \sqrt{R^2 \omega^2 + \alpha^2}} \\
\vec{a} &= (0 - R \omega^2) \hat{s} + (0 + 0) \hat{\phi} + 0 \hat{z} \\
\Aboxed{ &= -R \omega^2 \hat{s}.}
\end{aligned} \]

Notice how this makes sense because $a = \frac{v^2}{r} = \frac{R^2
\omega^2}{R}$
\end{proof}

\begin{ex}
Consider a mass of mass $m$ swinging on a rope of length $R$ with normal
gravity $g$. Assuming no friction, determine $\phi(t)$ where $\phi$
represents the angle from rest of the mass.

\begin{center}
\begin{asy}
size(6cm);

pair O = (0, 0);
pair P = (-1, -sqrt(3));

draw(arc(O, 2, 210, 330), Arrows);
draw("$R$", O -- P);
draw(O -- (0, -2), dashed);

draw("$\phi$", arc(O, 0.3, 240, 270));
draw("$g$", P -- (P + 0.8S), right, Arrow);

dot(O);
dot(P);

label("$m$", P, SW);
\end{asy}
\end{center}
\end{ex}

\begin{proof}[Solution]
Again we will refer back to the glorious equation $\vec{F} = m
\ddot{\vec{r}}$. Luckily $z = 0$ so we can leave that out of our
calculations.

\[ \vec{F} = mg(\cos \phi \hat{s} - \sin \phi \hat{\phi}) + \vec{F}_N \]
but $\vec{F}_N$ is such that $\frac{ds}{dt} = 0$ so we end up with
\[ \vec{F} = -mg \sin \phi \hat{\phi} = m \ddot{\vec{r}} \implies
\ddot{\vec{r}} = -g \sin \phi \hat{\phi}, \]
which we can rewrite as
\[ \left(s \ddot{\phi} + 2 \dot{s} \dot{\phi} \right) \hat{\phi} = -g
\sin \phi \hat{\phi} \implies \ddot{\phi} = -\frac{g}{s} \sin \phi, \]
which does not have a very nice solution, but we can approximate $\sin
\phi \approx \phi$ for small $\phi$, so this becomes $\ddot{\phi} =
-\frac{g}{s} \phi$, which has the nice solution
\[ \phi(t) = \alpha \sin\left( \sqrt{\frac{g}{s}} t + \psi \right) \]
for some optimal choice of $\psi$ and $\alpha = \max \phi$.
\end{proof}
