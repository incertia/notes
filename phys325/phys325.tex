\documentclass{article}

\usepackage[margin=1in]{geometry}
\usepackage{amsmath}
\usepackage{amsthm}
\usepackage{amsfonts}
\usepackage{amssymb}
\usepackage{fancyhdr}
\usepackage{parskip}

% setup the header
\pagestyle{fancy}
\lhead{Will Song}
\chead{Math 424}
\rhead{\today}

% setup definition/theorem etc
\newtheoremstyle{norm}
{3pt}
{3pt}
{}
{}
{\bf}
{:}
{.5em}
{}

\theoremstyle{norm}
\newtheorem{thm}{Theorem}[section]
\newtheorem{lem}[thm]{Lemma}
\newtheorem{df}[thm]{Definition}
\newtheorem{rem}[thm]{Remark}
\newtheorem{st}{Step}
\newtheorem{prop}[thm]{Proposition}
\newtheorem{cor}[thm]{Corollary}
\newtheorem{conj}[thm]{Conjecture}
\newtheorem{clm}[thm]{Claim}
\newtheorem{exr}[thm]{Exercise}
\newtheorem{ex}[thm]{Example}
\newtheorem{prb}[thm]{Problem}

% just useful shorthand
\renewcommand{\st}{\,\operatorname{s.t.}\,}
\let\hom\relax
\DeclareMathOperator{\hom}{Hom}
\DeclareMathOperator{\Tr}{Tr}
\DeclareMathOperator{\sgn}{sgn}
\DeclareMathOperator{\adj}{adj}

% for easier math
\everymath{\displaystyle}

\title{Math 424 Notes}
\author{Will Song}
\date{Fall 2015}

\begin{document}

\maketitle
\newpage

\tableofcontents
\newpage

\section{Day 1 - Set Theory and Cardinality}

Consult \emph{Introduction to Analysis} by M. Rosenlicht, Dover, 1986
for any info.

\begin{df}
A set $S$ is considered \textbf{infinite} if $\exists f : S \to
\mathbb{N}$ such that $f$ is surjective, and \textbf{finite} otherwise.
\end{df}

\begin{df}
A set $S$ is considered \textbf{countable} if $\exists f : S \to
\mathbb{N}$ such that $f$ is injective, and \textbf{uncountable}
otherwise.
\end{df}

\begin{df}
Denote by $\mathcal{P}(S)$ to be the \textbf{powerset} of $S$,
consisting of precisely all the subsets of $S$. More precisely,
\[ \mathcal{P}(S) = \lbrace X \mid X \subseteq S \rbrace. \]
\end{df}

\begin{rem}
Equivalently, $S$ is infinite if there exists injective $f : \mathbb{N}
\to S$ and  countable if there exists surjective $f : \mathbb{N} \to S$.
\end{rem}

\begin{ex}
$\mathcal{P}(\mathbb{N})$ is uncountable.
\end{ex}

\begin{proof}
Assume for the sake of contradiction that there exists a surjective $g :
\mathbb{N} \rightarrow \mathcal{P}(\mathbb{N})$.  Then for each $i \in
\mathbb{N}$, we have $g(i) \subseteq \mathbb{N}$.

We then take the set $A = \lbrace i \in \mathbb{N} : i \not \in g(i)
\rbrace$. By picking an arbitrary element $j \in \mathbb{N}$ such that
$g(j) = A$ by surjectivity, but $j \in A$ and $j \not \in A$ are both
false statements, so $g$ cannot exist.
\end{proof}

\begin{ex}
$\mathbb{R}$ is uncountable.
\end{ex}

\begin{proof}
Suppose it is not and take any surjection $g : \mathbb{N} \to
\mathbb{R}$. We take the natural isomorphism $\phi : \mathbb{R}
\rightarrow \mathbb{Z} \times X \times X \times \cdots$ where $X =
\mathbb{Z}/10\mathbb{Z}$. For example, $17.1333333\dots
\xrightarrow{\phi} (17, 1, 3, 3, 3, \dots)$. Now we have the enumeration
of $\mathbb{N}$ through $g$ expressed by $i \xrightarrow{g} (a_{i_1},
a_{i_2}, a_{i_3}, \dots)$ for some choices of $a_{i_j}$.

We now construct $x = (b_1, b_2, b_3, \dots)$ such that $b_i \neq
a_{i_i}, 0, 9$ which results in $x \neq g(i)$, $\forall i \in
\mathbb{N}$.
\end{proof}


\end{document}
