\documentclass{article}

\usepackage{incertia}

% setup the header
\pagestyle{fancy}
\lhead{Will Song}
\chead{Phys 325}
\rhead{HW3}

\begin{document}

\section{Homework 3}

\begin{prb}[Maximizing Momentum]
\label{prb1}
Consider a rocket with initial mass $m_0$, exhaust speed
$v^{\textrm{ex}}$, accelerating from rest in free space. For what value
of $m$ is $p$ maximized?
\end{prb}

\begin{proof}[Solution]
We refer to the rocket physics equation.
\[ m\dot{v} = -\dot{m}v^{\textrm{ex}} + F^{\textrm{ext}}, \]
but $F^{\textrm{ext}} = 0$ so
\[ m\dot{v} = -\dot{m}v^{\textrm{ex}}. \]
We now split into differentials to get
\[ \begin{aligned}
m \frac{dv}{dt} &= -\frac{dm}{dt}v^{\textrm{ex}} \\
m dv            &= -v^{\textrm{ex}} dm \\
dv              &= -\frac{v^{\textrm{ex}}}{m} dm. \\
\end{aligned} \]
Now we integrate to get momentum.
\[ \begin{aligned}
\int_0^v dv &= -\int_{m_0}^m \frac{v^{\textrm{ex}}}{m} dm \\
v &= v^{\textrm{ex}} \ln \frac{m_0}{m} \\
p = mv &= mv^{\textrm{ex}} \ln \frac{m_0}{m} \\
\end{aligned} \]
To maximize momentum, we must have $\frac{dp}{dm} = 0$, so we
differentiate in $m$ to get
\[ \frac{dp}{dm} = v^{\textrm{ex}}\left(\ln \frac{m_0}{m} - 1\right) \]
To get $\frac{dp}{dm} = 0$, we need $\ln \frac{m_0}{m} = 1$, or $m_0 =
me$. This rearranges to $\boxed{m = \frac{m_0}{e}}$.
\end{proof}

\begin{prb}[Staged Rockets]
A rocket with intial mass $m_0$ ejects fuel at $v^{\textrm{ex}}$ and
accelerates from rest in free space.
\begin{enumerate}[(a)]
\item
Suppose the rocket carries $\lambda = 80\%$ of its initial mass as fuel.
What is the final speed of the rocket after all the fuel is burned?
\item
Suppose the rocket carries $\lambda = 40\%$ of its fuel in each of its
two fuel tanks, each weighing $0.1 m_0$. The rocket jettisons the first
fuel tank upon completion and starts the second stage. Again, compute
the final speed once all the fuel is burned. Also compare this to part
(a).
\end{enumerate}
\textbf{Note}: A tank being jettisoned simply means that it is detached
(i.e. $v = 0$), and not ejected at some speed $v > 0$.
\end{prb}

\begin{proof}[Solution]
$ $
\begin{enumerate}[(a)]
\item
Again we refer to the rocket physics equation for free space.
\[ m\dot{v} = -\dot{m}v^{\textrm{ex}}. \]
In Problem~\ref{prb1}, we found that
\[ v = v^{\textrm{ex}} \ln \frac{m_0}{m}. \]
We set $m = 0.2 m_0$ to get
\[ \boxed{v = v^{\textrm{ex}} \ln 5 \approx 1.61 v^{\textrm{ex}}}. \]
\item
It's the same principle, but this time, we treat $v$ in the velocity
equation as a sort of $\Delta v$ and add them up for the two stages.
Hence, we get
\[ \boxed{v = \Delta v_1 + \Delta v_2 = v^{\textrm{ex}} \left( \ln
\frac{1}{0.6} + \ln \frac{0.5}{0.1} \right) \approx 2.12
v^{\textrm{ex}}} \]
Notice how this is faster than the speed we achieved in (a). In fact,
the second stage is precisely (a) but with half the mass.
\end{enumerate}
\end{proof}

\end{document}
