\documentclass{article}

\usepackage[margin=1in]{geometry}
\usepackage{amsmath}
\usepackage{amsthm}
\usepackage{amsfonts}
\usepackage{amssymb}
\usepackage{fancyhdr}
\usepackage{parskip}

% setup the header
\pagestyle{fancy}
\lhead{Will Song}
\chead{Math 424}
\rhead{\today}

% setup definition/theorem etc
\newtheoremstyle{norm}
{3pt}
{3pt}
{}
{}
{\bf}
{:}
{.5em}
{}

\theoremstyle{norm}
\newtheorem{thm}{Theorem}[section]
\newtheorem{lem}[thm]{Lemma}
\newtheorem{df}[thm]{Definition}
\newtheorem{rem}[thm]{Remark}
\newtheorem{st}{Step}
\newtheorem{prop}[thm]{Proposition}
\newtheorem{cor}[thm]{Corollary}
\newtheorem{conj}[thm]{Conjecture}
\newtheorem{clm}[thm]{Claim}
\newtheorem{exr}[thm]{Exercise}
\newtheorem{ex}[thm]{Example}
\newtheorem{prb}[thm]{Problem}

% just useful shorthand
\renewcommand{\st}{\,\operatorname{s.t.}\,}
\let\hom\relax
\DeclareMathOperator{\hom}{Hom}
\DeclareMathOperator{\Tr}{Tr}
\DeclareMathOperator{\sgn}{sgn}
\DeclareMathOperator{\adj}{adj}

% for easier math
\everymath{\displaystyle}

\title{Math 424 Notes}
\author{Will Song}
\date{Fall 2015}

\begin{document}

\maketitle
\newpage

\tableofcontents
\newpage

\section{Day 1 - Well Ordering}

\begin{prop}
Well ordering implies induction.
\end{prop}

\begin{proof}
Take $S \subseteq \mathbb{N}$ such that $1 \in S$ and $n + 1 \in S$ if
$n \in S$. Suppose for the sake of contradiction that $S \neq
\mathbb{N}$. Consider $S' = S \setminus \mathbb{N}$. By well ordering,
$S'$ has minimal element $m \neq 1$ due to $1 \in S$, so $m - 1 \in S$
by set difference. But then $m \in S$ by assumption on $S$ so $m \not
\in S'$ so $S'$ has no minimal element which contradicts well ordering.
\end{proof}

\begin{prop}
Induction implies well ordering
\end{prop}

\begin{proof}
Suppose $S \subseteq \mathbb{N}, S \neq \varnothing$.

Suppose that $S$ has no smallest element. Then $1 \not \in S$ because
$1$ is the minimal element of $\mathbb{N}$ and thus $S$. Suppose $1, 2,
3, \dots, k \not \in S$, then $k + 1 \not \in S$ because it would then be
the smallest element in $S$. Then $n \not \in S \forall n \in
\mathbb{N}$, so $S = \varnothing$.
\end{proof}

\begin{prop}
For any $a, d \in \mathbb{Z}, d \neq 0$, then we can write $a = qd + r,
0 \leq r < d$ with $q, d$ unique.
\end{prop}

\begin{proof}
% cheating
$ $\newline
\textbf{Existence}: Consider the set $S = \lbrace x = a - qd \mid x
\geq 0 \rbrace$. At least one of $a$, $-a$ is in $S$ so $S$ is non-empty
and has a minimal element by well ordering. We take $r = \min S$ and $q$
to be the corresponding $r = a - qd$. If $r \not < d$, then $r - d = a -
(q + 1)d \in S$ which contradicts the minimality of $r$, so we are done
(for existence).

\textbf{Uniqueness}: Let $a = q_1 d + r_1 = q_2 d + r_2$. Then $0 = a -
a = (q_1 - q_2) d + (r_1 - r_2)$, or $q_1 - q_2 = 0, r_1 - r_2 = 0$
because of restrictions on $r_1, r_2$.
\end{proof}


\end{document}
