\documentclass{article}

\usepackage[margin=1in]{geometry}
\usepackage{amsmath}
\usepackage{amsthm}
\usepackage{amsfonts}
\usepackage{amssymb}
\usepackage{fancyhdr}
\usepackage{parskip}

% setup the header
\pagestyle{fancy}
\lhead{Will Song}
\chead{Math 416}
\rhead{\today}

% setup definition/theorem etc
\newtheoremstyle{norm}
{3pt}
{3pt}
{}
{}
{\bf}
{:}
{.5em}
{}

\theoremstyle{norm}
\newtheorem{thm}{Theorem}[section]
\newtheorem{lem}[thm]{Lemma}
\newtheorem{df}[thm]{Definition}
\newtheorem{rem}[thm]{Remark}
\newtheorem{st}{Step}
\newtheorem{prop}[thm]{Proposition}
\newtheorem{cor}[thm]{Corollary}
\newtheorem{conj}[thm]{Conjecture}
\newtheorem{clm}[thm]{Claim}
\newtheorem{exr}[thm]{Exercise}
\newtheorem{ex}[thm]{Example}
\newtheorem{prb}[thm]{Problem}

% just useful shorthand
\renewcommand{\st}{\,\operatorname{s.t.}\,}
\let\hom\relax
\DeclareMathOperator{\hom}{Hom}
\DeclareMathOperator{\Tr}{Tr}
\DeclareMathOperator{\sgn}{sgn}
\DeclareMathOperator{\adj}{adj}

% for easier math
\everymath{\displaystyle}

\title{Math 416 Notes}
\author{Will Song}
\date{Fall 2014}

\begin{document}

\maketitle
\newpage

\tableofcontents
\newpage

\section{Definitions}
This section is reserved for defining what certain things are and for
propositions with very short proofs.

\subsection{Groups}
\begin{df}
A \textbf{semi-group} is a set $G$ with an associative binary operation $* : G
\times G \rightarrow G$.
\end{df}

\begin{df}
A semigroup $G$ is a \textbf{monoid} if we can find an identity $1 \in G \, \st
\, \forall g \in g, g * 1 = 1 * g = g$.
\end{df}

\begin{prop}
The identity is unique.
\end{prop}
\begin{proof}
Suppose $1, 1' \in G$ are both identities. Then
\[ 1 = 1 * 1' = 1'. \]
\end{proof}

\begin{df}
A set $G$ with an associative binary operation $* : G \times G
\rightarrow G$ is called a \textbf{group} if it satisfies the following
properties:
\begin{itemize}
\item $\forall g \in G, \exists 1 \in G \st g * 1 = 1 * g = g$
\item $\forall g \in G, \exists g^{-1} \in G \st g * g^{-1} = g^{-1} * g
= 1$
\end{itemize}
\end{df}
\begin{rem}
A group is equivalent to a monoid with inverses.
\end{rem}
\begin{prop}
Inverses are unique.
\end{prop}
\begin{proof}
Let $x_1, x_2$ both be inverses of $g \in G$. Then
\[ x_1 = x_1 * g * x_2 = x_2. \]
\end{proof}

\begin{df}
A group $G$ is called abelian or commutative if and only if $\forall a,
b \in G, a * b = b * a$.
\end{df}

\begin{df}
Let $G, H$ be groups. A \textbf{group homomorphism} is a map $f : G
\rightarrow H$ that preserves multiplication. In other words,
\[ f(ab) = f(a) f(b) \quad \forall a, b \in G. \]
\end{df}

\begin{df}
Given two sets $X, Y$, we define their direct product to be
\[ X \times Y = \left\lbrace \left(x, y\right) : x \in X, y \in Y
\right\rbrace. \]
\end{df}

\begin{df}
Let $\varphi : G \rightarrow H$ be a group homomorphism. Define the
\textbf{kernel} of $\varphi$ to be
\[ \ker \varphi = \left\lbrace g \in G : \varphi(g) = 1 \right\rbrace \]
where $1$ is the identity in $H$.
\end{df}

\begin{df}
Let $G$ be a semigroup/monoid/group. We say $H \leq G$ is a
\textbf{sub-semigroup} if $h_1, h_2 \in H \implies h_1 h_2 \in H$. If
$G$ is monoid and $1 \in H$, then $H$ is a \textbf{sub-monoid}. Finally,
if $H$ is closed under multiplication, $H$ is a \textbf{sub-group}.
\end{df}

\begin{df}
A \textbf{left coset} of a subgroup $H$ of $G$ with respect to an element $g
\in G$ is the set
\[ gH = \left\lbrace gh : h \in H \right\rbrace. \]
A \textbf{right coset} is defined similarly, but with multiplication on
the right.
\end{df}

\begin{df}
An \textbf{equivalence relation} $\sim$ is something that is symmetric,
reflexive, and transitive. For example, $=$ for the set of integers.
\end{df}
\begin{prop}
An equivalence relation $\sim$ partitions a set $X$ into cosets if $X$
is a group or more generally, equivalence classes.
\end{prop}
\begin{proof}
For any subgroup $H$, we define the equivalence relation $\sim$ on $X$
to be $a \sim b$ if $a = bh$ for some $h \in H$. \\
\textbf{Reflexivity:} $a = ah$ for $h = 1 \in H$ because $H$ is a
subgroup. \\
\textbf{Symmetry:} If $a = bh$, then $b = ah^{-1}$, but $H$ is a
subgroup so $h^{-1} \in H$ is true, so $b \sim a$ as well. \\
\textbf{Transitivity:} Suppose $a \sim b$ and $b \sim c$. Then $a =
bh_1$ and $b = ch_2$, so $a = ch_2h_1$, but $h_2 h_1 \in H$ because $H$
is a subgroup. \\
So the equivalence classes of this equivalence relation are precisely
the left cosets of $X$. Whether or not this works for any other
equivalence relation, if there are any, I do not know.
\end{proof}
\begin{rem}
Because of the previous proposition, it makes sense to speak of the
quotient $X/\sim$ (the equivalence classes of $X$ under $\sim$) as well
as define a mapping $\pi : X \rightarrow X/\sim$ that takes any element
$x \in X$ to its coresponding equivalence class.
\end{rem}

\begin{df}
Call a subgroup $H \leq G$ normal if $\forall h \in H, g \in G$ we have
\[ ghg^{-1} \in H. \]
\end{df}
\begin{rem}
Any abelian subgroup is normal.
\end{rem}
\begin{prop}
$\ker \varphi$ is normal for a homomorphism $\varphi$
\end{prop}
\begin{proof}
We begin by showing that $\ker \varphi$ is a subgroup. \\
\textbf{Identity:} $\varphi(a) = \varphi(1a) = \varphi(1) \varphi(a)$ so
$1 \in \ker \varphi$ \\
\textbf{Inverse:} Suppose $a \in \ker \varphi$. We wish to show that
$a^{-1} \in \ker \varphi$, or $\varphi(a^{-1}) = 1$. But this is easy
because $1 = \varphi(1) = \varphi(aa^{-1}) = \varphi(a) \varphi(a^{-1})
= 1 \varphi(a^{-1})$. \\
\textbf{Associativity:} Suppose $a, b, c \in \ker \varphi$. Then,
\[ \varphi(a(bc)) = \varphi(a)(\varphi(b)\varphi(c)) =
(\varphi(a)\varphi(b))\varphi(c) = \varphi((ab)c) \]
for the silly reason that $\varphi(a) = \varphi(b) = \varphi(c) = 1$. \\
Now we can do normality. \\
\textbf{Normality:} Suppose $h \in \ker \varphi$. Then
\[ \varphi(ghg^{-1}) = \varphi(g)\varphi(h)\varphi(g^{-1}) = \varphi(g)
\varphi(g^{-1}) = \varphi(g g^{-1}) = \varphi(1) = 1, \]
so $ghg^{-1} \in \ker \varphi$ for all $g \in G$.
\end{proof}

\subsection{Rings}
\begin{df}
A \textbf{ring} is a set $R$ combined with two binary operations $+$ and
$*$ such that,
\begin{itemize}
\item $R$ is an abelian group under $+$ with identity $0$. The inverse
of $r \in R$ is usually denoted as $-r$.
\item $R$ is a monoid under $*$ with identity $1$
\item $*$ is distributive over $+$
\end{itemize}
\end{df}

\begin{df}
A ring is called commutative if multiplication in the ring is
commutative.
\end{df}

\begin{prop}
Let $R$ be a ring. Then $0 * r = 0 = 0 * r \quad \forall r \in R$.
\end{prop}
\begin{proof}
We compute
\[ 0 * r = \left(0 + 0\right) * r = 0 * r + 0 * r. \]
We add the inverse to get $0 * r = 0$. The right hand equality is done
in a similar fashion.
\end{proof}

\begin{prop}
Let $R$ be a ring. Then $\left(-1\right) * r = -r = r * \left(-1\right)$.
\end{prop}
\begin{proof}
Again, compute
\[ \left(-1\right) * r + r = \left(-1\right) * r + 1 * r = \left(-1 +
1\right) * r = 0 * r = 0. \]
We get the desired result by adding the inverse of $r$. The same can be
done with the right hand equality.
\end{proof}

\begin{df}
A commutative ring $R$ is called a \textbf{field} if $R \backslash \lbrace 0
\rbrace $ is a group with respect to multiplication.
\end{df}

\begin{df}
We can also have a \textbf{ring homomorphism} $\varphi : R \rightarrow
S$ which preserves both addition and multiplication. Most importantly,
we should also have $\varphi(1) = 1$.
\end{df}

\subsection{Modules}
\begin{df}
A \textbf{left $R$-module} is an additive group $M$ (meaning $M$ is an
abelian group with operation $+$) with an additional left multiplication
by elements in $R$, which shall be called \textbf{scalar
multiplication}, denoted here by $\cdot : R \times M \rightarrow M$,
such that
\begin{itemize}
\item $r_1 \cdot \left(r_2 \cdot m\right) = \left(r_1 * r_2\right) \cdot
m \quad \forall r_1, r_2 \in R, m \in M$
\item $\left(r_1 + r_2\right) \cdot m = r_1 \cdot m + r_2 \cdot m \quad
\forall r_1, r_2 \in R, m \in M$ and $r \cdot \left(m_1 + m_2\right) = r
\cdot m_1 + r \cdot m_2 \quad \forall r \in R, m_1, m_2 \in M$.
\item $1 \cdot m = m \quad \forall m \in M$.
\end{itemize}
\end{df}

\begin{df}
A \textbf{right $R$-module} is defined similarly, except $R$ acts on $M$
on the right.
\end{df}

\begin{df}
If $M$ is a left $R$-module and a right $S$-module, then we call it an
\textbf{$R$-$S$-bimodule} if it satisfies
\[ r\left(ms\right) = \left(rm\right)s. \]
\end{df}
\begin{rem}
An $R$-$R$-bimodule is also called an $R$-bimodule.
\end{rem}

\begin{rem}
For a left $R$-module $M$, we can define an analogous right $R$-module
$M$ by assigning $m \cdot r = rm$. Notice that this gives
\[ \begin{aligned}
\left(r_1 r_2\right) m &= r_1 \left(r_2 m\right) \\
&= \left(m r_2\right) r_1 \\
&= m \left(r_2 r_1\right) \\
&\neq m \left(r_1 r_2\right)
\end{aligned} \]
in general. The first and last expressions are equal if and only if $R$
is a commutative ring. If this is the case, we just say that $M$ is an
$R$-module.
\end{rem}
\begin{rem}
Notice that $R$ is also a $R$ module, where scalar multiplication is
just multiplication in $R$.
\end{rem}

\begin{cor}
All abelian groups are $\mathbb{Z}$-modules.
\end{cor}

\begin{df}
We can also have an \textbf{$R$-module homomorphism} $\varphi : M
\rightarrow N$ that preserves addition in $M$ and the action of $R$.
That is, $\varphi(m_1 + m_2) = \varphi(m_1) + \varphi(m_2)$ and
$\varphi(rm) = r \varphi(m)$.
\end{df}

\begin{df}
Let $M_1, M_2$ be $R$-modules. Define the direct sum of $M_1$ and $M_2$,
denoted $M_1 \oplus M_2$, to be the module whose underlying group is
$M_1 \times M_2$. The action of $R$ on $M_1 \times M_2$ is done
component-wise. \\
We can also generalize this to
\[ R \oplus R \oplus \cdots \oplus R = R^{\oplus n} \]
where there are $n$ $R$s on the left hand side.
\end{df}

\subsection{Permutations}
\begin{df}
A \textbf{permutation} $\sigma$ is a bijection of the set $\lbrace 1, 2,
\dots, n \rbrace$ to itself.
\end{df}

\begin{df}
A \textbf{transposition} is a permutation $\sigma$ that satisfies
$\sigma(i) = j, \sigma(j) = i$, and $\sigma(k) = k \quad \forall k \neq
i, j$ for some $i, j$.
\end{df}

\begin{rem}
$\sigma(\sigma(a)) = a$ when $\sigma$ is a transposition.
\end{rem}

\begin{prop}
Every permutation is a composition of transpositions.
\end{prop}

\begin{proof}
Every permutation can be represented as a product of \textbf{cycles},
and every cycle can be represented as a product of transpositions.
\end{proof}

\begin{prop}
We define the \textbf{signum}, or \textbf{sign}, of a permutation to be
\[ \sgn(T_n \cdots T_1) = \sgn(\sigma) = \prod_{a < b} \frac{\sigma(b) -
\sigma(a)}{b - a} = (-1)^n, \]
where $T_i$ represents a transposition. We also claim that
$\sgn(\sigma_1 \sigma_2) = \sgn(\sigma_1) \sgn(\sigma_2)$.
\end{prop}

\begin{proof}
It's obvious that the product is either $1$ or $-1$, because it's the
product of the same terms with possibly different signs and possibly
different order. We only need to prove the multiplicativity for
transpositions because every permutation is just a product of
transpositions.
\[ \begin{aligned}
\sgn(\sigma_1 \sigma_2) &= \prod \frac{\sigma_1(\sigma_2(b)) -
\sigma_1(\sigma_2(a))}{b - a} \\
&= \prod \frac{\sigma_1(d) - \sigma_1(c)}{\sigma_2(d) - \sigma_2(c)} \\ 
&= \prod \frac{\sigma_1(d) - \sigma_1(c)}{d - c} \prod \frac{d -
c}{\sigma_2(d) - \sigma_2(c)} \\
&= \sgn(\sigma_1) \sgn(\sigma_2)
\end{aligned} \]
where $\sigma_1$ can be an arbitrary permutation and $\sigma_2$ is an
arbitrary transposition. This finishes the proof.
\end{proof}

\begin{df}
We will let $S_n$ denote the set of permutations on the set $\lbrace 1,
\dots, n \rbrace$.
\end{df}

\section{More Modules}
This section is reserved for some crucial propositions regarding
modules.

\subsection{Homomorphisms With Modules}

\begin{df}
We will use the notation $\hom(X, Y)$ to denote the set of all
homomorphisms from $X$ to $Y$.
\end{df}

\begin{prop}
Let $\varphi$ be a homomorphism. Then $\varphi$ is injective if and only
if $\ker \varphi = \lbrace 0 \rbrace$.
\end{prop}
\begin{proof}
We first prove that $\varphi$ is injective if $\ker \varphi = \lbrace 0
\rbrace$. Suppose $\varphi(\alpha) = \varphi(\beta)$. Then
$\varphi(\alpha) + (-\varphi(\beta)) = 0$, or $\varphi(\alpha) +
\varphi(-\beta) = 0 \iff \varphi(\alpha + (-\beta)) = 0$. However, $\ker
\varphi = \lbrace 0 \rbrace$ so $\alpha + (-\beta) = 0$, or $\alpha =
\beta$. \\
Now for the other direction. Suppose $\varphi$ is injective. We have
that $\varphi(\alpha) = \varphi(0 + \alpha) = \varphi(0) +
\varphi(\alpha) \implies \varphi(0) = 0$, so $\varphi(\alpha) = 0 =
\varphi(0) \implies \alpha = 0$ by injectivity, so we are done.
\end{proof}

\begin{prop}
Take a $R$-module $M$. The map $\chi : \hom_R(R, M) \rightarrow M$ by
$\chi(\varphi) = \varphi(1)$, or $\varphi \mapsto \varphi(1)$ is an
\textbf{isomorphism}, or a homomorphism that admits an inverse. In other
words, a homomorphism from a ring $R$ to a $R$-module is determined by
its value at $1$.
\end{prop}
\begin{proof}
Let $\varphi, \psi \in \hom_R(R, M)$ and $r \in R$. We can compute that
\[ \chi(\varphi + r \psi) = (\varphi + r \psi)(1) = \varphi(1) + r
\psi(1) = \chi(\varphi) + r \chi(\psi), \]
which gives that $\chi$ is a $R$-module homomorphism. \\
Now suppose $\varphi = \ker \chi$. Then $\forall r \in R$, 
\[ \varphi(r) = r \varphi(1) = r \chi(\varphi) = r * 0 = 0, \]
implying $\varphi$ is injective. \\
Now suppose $m \in M$. We can define a $\psi_m : R \rightarrow M$ by
$\psi_m(r) = rm$. We can easily compute that
\[ \psi_m(r_1 + k r_2) = r_1 m + (k r_2) m = r_1 m + k (r_2 m) =
\psi_m(r_1) + k \psi_m(r_2), \]
so $\psi_m$ is a $R$-module homomorphism, meaning $\psi_m \in \hom_R(R,
M)$. Moreover, $\chi(\psi_m) = \psi_m(1) = 1m = m$ so $\chi$ is
surjective. \\
Because $\chi$ is a homomorphism and is bijective, it is an isomorphism
and we are done.
\end{proof}

% \begin{prop}
% We also have the following bijections:
% \[ \hom_R(X_1 \cup X_2, Y) \leftrightarrow \hom_R(X_1, Y) \times
% \hom_R(X_2, Y) \]
% and
% \[ \hom_R(X, Y_1 \times Y_2) \leftrightarrow \hom_R(X, Y_1) \times
% \hom_R(X, Y_2) \]
% \end{prop}
% \begin{proof}
% I will include an example, taken from Evan Chen. Suppose you want to
% define a function on the set $\lbrace 1, 2, 3 \rbrace$. It is the same
% as defining a function on the set $\lbrace 1, 2 \rbrace$ and then
% another function on the set $\lbrace 3 \rbrace$. \\
% Basically, you take two projections $\alpha_1, \alpha_2$ into $X_1
% \sqcup X_2$ and the compose them with your function from $X_1 \sqcup
% X_2$ to $Y$. \\
% Seeing the logic behind the 2nd is left as an exercise to the reader.
% \end{proof}
% 
% \begin{prop}
% We can extend this to direct sums as well. Take any two $R$-modules
% $M_1, M_2$. Then there exists a bijection
% \[ \hom_R(M_1 \oplus M_2, Y) \leftrightarrow \hom_R(M_1, Y) \times
% \hom_R(M_2, Y) \]
% and
% \[ \hom_R(X, M_1 \oplus M_2) \leftrightarrow \hom_R(X, M_1) \times
% \hom_R(X, M_2) \]
% \end{prop}
% 
% \begin{rem}
% We can now consider larger direct sums $M_1, M_2, \dots, M_n$. However,
% we are more interested in the bijection
% \[ \hom_R(R^{\oplus n}, M) \leftrightarrow \hom_R(R, M)^n \simeq M^n \]
% \end{rem}
% 
% \begin{prop}
% Let $A$ be an indexing set, and consider the set of modules $(M_a)_{a
% \in A}$. Then
% \[ M = \prod_{a \in A} M_a \]
% also has a module structure.
% \end{prop}
% 
% \begin{df}
% We define the infinite direct sum $\bigoplus_a M_a$, which is a subset
% of $\prod_a M_a$, where all but finitely many indices $A$ is zero.
% \end{df}


\section{Vector Spaces and Related Things}
\subsection{Vector Spaces}
We should now have enough abstract definitions to introduce vector
spaces in an abstract enough way. I am purposefully trying to stay away
from commutative maps because they confuse me and learning linear
algebra at such an abstract level is confusing enough for me.

\begin{df}
A \textbf{vector space} $V$ over a field $F$ is an abelian group under
$+$ and an $F$-module.
\end{df}

\begin{df}
The \textbf{span} of a set of vectors $v_1, v_2, \dots, v_k$ is defined
as the set
\[ \left\lbrace \sum_i a_i v_i : a_i \in F \right\rbrace. \]
Similarly, a set of vectors span a set of every element of that set can
be written as a linear combination of the set of vectors.
\end{df}

\begin{df}
A set of vectors $v_1, \dots, v_k$ are said to be \textbf{linearly
independent} if $\sum_i a_i v_i = 0 \Leftrightarrow a_i = 0 \quad
\forall i$.
\end{df}

\begin{rem}
Notice that this guarantees that each decompisition into the linearly
independent vectors is unique.
\end{rem}

We now prove a fundamental lemma regarding linearly independent,
spanning sets.
\begin{lem}[Steinitz Exchange Lemma]
Let $V$ be a vector space. Let $v_1, v_2, \dots, v_k$ be a set of
linearly independent vectors in $V$. Let $w_1, w_2, \dots, w_m$ span
$V$. We then have $k \leq m$ and (possibly after reordering $w_i$), the
set of vectors $v_1, \dots, v_k, w_{k + 1}, \dots, w_m$ also span $V$.
\end{lem}

\begin{proof}
We induct on $k$ to get our result. The case $k = 0$ is obvious, as we
don't need to replace anything and the statement of the lemma solves
itself. So suppose the lemma holds positive for some $k < m$. Then,
possibly after reordering, the vectors $v_1, v_2, \dots, v_k, w_{k + 1},
\dots, w_m$ span $V$. We can then write
\[ v_{k + 1} = \sum_i \alpha_i v_i + \sum_i \alpha_i w_i. \]
At least one $\alpha_j$ must be non-zero, so we immediately get $k < m$,
or $k + 1 \leq m$. We can now ``solve'' for $w_{k + 1}$ to get
\[ w_{k + 1} = \frac{1}{\alpha_{k + 1}} \left(v_{k + 1} - \sum_{i =
1}^{k} \alpha_i v_i - \sum_{i = k + 2}^{m} \alpha_i w_i\right). \]
Hence $w_{k + 1}$ is in the span of the set $\lbrace v_1, \dots, v_{k +
1}, w_{k + 2}, \dots, w_m \rbrace$, which means that everything in the
span of $\lbrace v_1, \dots, v_k, w_{k + 1}, \dots, w_m \rbrace$ is also
in the span of $\lbrace v_1, \dots, v_{k + 1}, w_{k + 2}, w_m \rbrace$,
finishing the proof.
\end{proof}

\begin{df}
The cardinality of a set of linearly independent vectors that span a
vector space $V$ is called the \textbf{dimension} of $V$, denoted
$\dim(V)$.

This is also the cardinality of the largest set of linearly independent
vectors in $V$. This is easily seen by the above lemma.
\end{df}

\begin{df}
We say a set of vectors is a \textbf{basis} for a vector space $V$ if
the set contains linearly independent vectors and spans $V$.
\end{df}

\begin{df}
We say $W$ is a \textbf{subspace} of $V$ over $F$ if and only if $W$
satisfies:
\begin{itemize}
\item $0 \in W$
\item $w_1 + w_2 \in W \quad \forall w_1, w_2 \in W$
\item $k w \in W \quad \forall w \in W, k \in F$.
\end{itemize}
\end{df}

Anyways, here are some examples of vector spaces and subspaces
\begin{itemize}
\item Any field $F$, such as $\mathbb{R}$ or $\mathbb{C}$
\item Any repeated direct sum of a field $F$, such as $\mathbb{R}^3$
\item The space of functions on $\mathbb{R}$
\begin{itemize}
\item Odd functions on $\mathbb{R}$
\item Even functions on $\mathbb{R}$
\end{itemize}
\end{itemize}

\begin{prop}
Let $V, W$ be subspaces. Then
\[ \dim(V) + \dim(W) = \dim(V + W) + \dim(V \cap W) \]
where $V + W = \lbrace v + w : v \in V, w \in W \rbrace$.
\end{prop}

\begin{proof}
Let $\alpha_1, \dots, \alpha_n$ be a basis for $V \cap W$. We can then
find the basis $\alpha_1, \dots, \alpha_n, v_1, \dots, v_k$ for $V$ and
the same for $W$, $w_1, \dots, w_l, \alpha_1, \dots, \alpha_n$. If we
can prove that $\alpha_1, \dots, \alpha_n, v_1, \dots, v_k, w_1, \dots,
w_l$ are a basis for $V + W$, we are done because
\[ (n + k) + (n + l) = (n + k + l) + n \]
It is obvious that they span $V + W$ by the definition of span, so it
suffices to show that they are linearly independent. Suppose that
\[ 0 = \sum a_i \alpha_i + \sum b_i v_i + \sum c_i w_i \]
We rearrange this equation to arrive at
\[ -\sum a_i \alpha_i = \sum b_i v_i + \sum c_i w_i, \]
implying $-\sum a_i \alpha_i, \sum b_i v_i + \sum c_i w_i \in V \cap W$.
However, $v_i$ and $w_i$ are not in the set $V \cap W$, otherwise they
would not be linearly independent with the $\alpha_i$, which implies
that $b_i = c_i = 0$ for all $i$, otherwise the RHS would not be in the
set $V \cap W$. Now use the fact that $\alpha_i$ form a basis to get
that $a_i = 0$, so we are done.
\end{proof}

\begin{prop}
A vector space $V$ over a field $F$ with dimension $d$ is isomorphic to
$F^d$.
\end{prop}

\begin{proof}
We just choose a basis and express each vector in $V$ as the
coefficients in the expansion as a sum of linearly independent vectors.
\end{proof}

For example, consider the space of polynomials with complex coefficients
with degree less than or equal to $3$. We will call this space
$\mathcal{P}_3$. Then the vector $1 + (3 - 2i)x + 7x^2$ is the same as
$(1, 3 - 2i, 7, 0)$ with the basis of $\lbrace 1, x, x^2, x^3 \rbrace$.

\begin{prb}
Determine a necessary and sufficient condition for a polynomial $P \in
\mathbb{R}[X]$ such that $P(x) \in \mathbb{Z} \forall x \in
\mathbb{Z}$.
% TODO: Actually prove this thing
\end{prb}

\begin{proof}[Solution]
I claim that $P$ must be an integral linear combination of the
polynomials $\binom{X}{k} \forall k \in \mathbb{N}$.
\end{proof}

\subsection{Linear Maps}
\begin{df}
Let $V, W$ be vector spaces.  A map $L : V \rightarrow W$ is said to be
\textbf{linear} if it satisfies:
\begin{itemize}
\item $L(v_1 + v_2) = L(v_1) + L(v_2) \quad \forall v_1, v_2 \in V$
\item $L(c v) = c L(v) \quad \forall c \in F, v \in V$
\end{itemize}
\end{df}

\begin{df}
The \textbf{nullspace} of a linear map $L$ is the same as the kernel of
$L$. We will denote the nullspace with $\mathcal{N}(L)$.
\end{df}

\begin{df}
The \textbf{range} of $L$ is simply the set $\mathcal{R}(L) = \lbrace
L(v) : v \in V \rbrace$.
\end{df}

\begin{prop}
Let $V, W$ be vector spaces and let $L : V \rightarrow W$ be linear.
Then $L$ is injective if and only if $\mathcal{N}(L) = \lbrace 0
\rbrace$.
\end{prop}

\begin{proof}
The forward direction is trivial because $L(v) = 0 = L(0) \implies v =
0$.

Going backwards, suppose $L(v_1) = L(v_2)$. Then, we have
\[ \begin{aligned}
L(v_1) - L(v_2) &= 0 \\
L(v_1 - v_2) &= 0 \\
v_1 - v_2 &= 0 \\
v_1 &= v_2
\end{aligned} \]
by linearity in $L$ and the fact that $v_1 - v_2 \in \mathcal{N}(L)$.
\end{proof}

\begin{thm}[Rank-Nullity Theorem]
Let $V, W$ be finite dimensional vector spaces and let $L : V
\rightarrow W$ be linear. Then
\[ \dim(\mathcal{N}(L)) + \dim(\mathcal{R}(L)) = \dim(V). \]
\end{thm}

\begin{proof}
Let $\lbrace \alpha_i \rbrace_{i = 1}^{k}$ be a basis for
$\mathcal{N}(L)$. By the exchange lemma, we also have $v_{k + 1}, \dots,
v_n$ so that $\alpha_i, v_i$ form a basis for $V$. If we can show that
$\dim(\mathcal{R}(L)) = n - k$, we are done, so we claim that $L(v_{k +
1}), \dots, L(v_n)$ form a basis for $\mathcal{R}(L)$.

\textbf{Span}: Suppose $z \in \mathcal{R}(L)$. Then we have
\[ \begin{aligned}
z &= L(v) \\
z &= L\left(\sum a_i \alpha_i + \sum b_i v_i\right) \\
z &= \sum a_i L(\alpha_i) + \sum b_i L(v_i) \\
z &= \sum b_i L(v_i) \implies z \in \operatorname{span}(L(v_i))
\end{aligned} \]
by linearity in $L$ and the stupid reason that $L(\alpha_i) = 0$.

\textbf{Linear independence}: Suppose $0 = \sum a_i L(v_i)$. Then
$L\left(\sum a_i v_i\right) = 0 \implies \sum a_i v_i \in \mathcal{N}(L)
\implies \sum a_i v_i = \sum b_i \alpha_i$, or $\sum a_i v_i + \sum -b_i
v_i = 0$, giving $a_i, b_i = 0$.
\end{proof}

\begin{rem}
$\dim(\mathcal{N}(L))$ is usually called the nullity of $L$ while
$\dim(\mathcal{R}(L))$ is usually called the rank of $L$, hence the name
of the theorem.
\end{rem}

\subsection{Matrices of Linear Maps}
Because linear maps are linear, they are given completely by what they
do on a basis. This motivates us to find some sort of compact
representation for a linear map.

\begin{df}
The \textbf{matrix} of a linear map $L : V \rightarrow W$ is defined as
follows. We choose a basis for $V$ and $W$, and the columns of the
matrix of $L$ is the coordinate decomposition of $L$ on the basis of
$V$.
\end{df}

\begin{ex}
The matrix of the linear transforomation in $\mathbb{R}^3$ to
$\mathbb{R}^3$ that takes $(1, 0, 0), (0, 1, 0), (0, 0, 1)$ to $(1, 2,
3),  (4, 5, 6), (7, 8, 9)$ would be represented as
\[ \begin{pmatrix}
1 & 4 & 7 \\
2 & 5 & 8 \\
3 & 6 & 9
\end{pmatrix} \]
\end{ex}

\begin{ex}
Consider the polynomial space $\mathcal{P}_3$ and let the linear
transformation be differentiation. The matrix of differentiation (taking
$\mathcal{P}_3$ to $\mathcal{P}_2$) with respect to $1, x, x^2, x^3$ and
$1, x, x^2$ would be
\[ \begin{pmatrix}
0 & 1 & 0 & 0 \\
0 & 0 & 2 & 0 \\
0 & 0 & 0 & 3
\end{pmatrix} \]
\end{ex}

We can then easily compute the transformation of any vector once bases
are chosen. This is done by "dotting", or multiplying each row vector's
components with the components of the column vector representation of
the vector being transformed, and then summing up the multiplied
components.

\begin{ex}
\[ \begin{pmatrix}
0 & 1 & 0 & 0 \\
0 & 0 & 2 & 0 \\
0 & 0 & 0 & 3
\end{pmatrix} \begin{pmatrix}
1 \\ 2 \\ 3 \\ 4
\end{pmatrix} = \begin{pmatrix}
2 \\ 6 \\ 12
\end{pmatrix}, \]
which is the same as saying $\frac{d}{dx}(1 + 2x + 3x^2 + 4x^3) = 2 + 6x
+ 12x^2$.
\end{ex}

\begin{rem}
We can compose linear transformations. If $T_1 : U \rightarrow V, T_2 :
V \rightarrow W$, then we can compute $T_3 : U \rightarrow W$ with $T_3
= T_2 T_1$, which is defined as $T_3(u) = T_2(T_1(u))$. Computing the
matrix of the resulting transform is also simple, and this is what
defines \textbf{matrix multiplication}. You basically transform each
column vector in the matrix of $T_1$ by $T_2$ and set that as the column
vector in the respective column of $T_3$. If this does not make sense,
experiment a little bit and see why this should be true.
\end{rem}

\begin{rem}
Matrix multiplication is associative. The easiest way to see this is by
going back to composition of transformations. $T_3 (T_2 T_1) = (T_3 T_2)
T_1$ simply because $T_3 (T_2 T_1) u = T_3(T_2(T_1(u))) = (T_3 T_2) T_1
u$.
\end{rem}

\begin{rem}
$A_{ij}$ will denote the $i,j$-th entry of a matrix $A$, or the entry in
the $i$-th row and $j$-th column.
\end{rem}

\section{Square Matrices}
Square matrices are very special, so they get their own section.


\section{Moving Towards Concreteness}
A lot of the things here will be restricted to the fields $\mathbb{F}
\in \lbrace \mathbb{C}, \mathbb{R}, \overline{\mathbb{Q}} \rbrace $.

\subsection{I'm In Space}
Because we needed more spaces in linear algebra.

\begin{df}
A \textbf{inner product space} $V$ is a vector space with an additional
structure called the \textbf{inner product} $\langle \cdot, \cdot
\rangle : V \times V \rightarrow \mathbb{F}$ satisfying
\begin{itemize}
\item \textbf{Conjugate symmetry} or \textbf{Hermitian symmetry}:
$\langle x, y \rangle = \overline{\langle y, x \rangle}$.
\item Linearity in the first argument: $\langle ax, y \rangle = a
\langle x, y \rangle$ and $\langle x + y, z \rangle = \langle x, z
\rangle + \langle y, z \rangle$.
\item \textbf{Positive definiteness}: $\langle x, x \rangle \geq 0$ and
$\langle x, x \rangle = 0 \implies x = 0$.
\end{itemize}
\end{df}

\begin{rem}
Because of conjugate symmetry, we have
\begin{enumerate}
\item $\langle x, cy \rangle = \overline{\langle cy, x \rangle} =
\overline{c} \overline{\langle y, x \rangle} = \overline{c} \langle x, y
\rangle$.
\item $\langle x, y + z \rangle = \overline{\langle y + z, x \rangle} =
\overline{\langle y, x \rangle} + \overline{\langle z, x \rangle} =
\langle x, y \rangle + \langle x, z \rangle$.
\end{enumerate}
\end{rem}

\begin{df}
A \textbf{metric space} is a set $M$ in which there is a metric
(distance) $d : M \times M \rightarrow \mathbb{R}$ such that
\begin{itemize}
\item $d(x, y) \geq 0$
\item $d(x, y) = 0 \Leftrightarrow x = y$
\item $d(x, y) = d(y, x)$
\item $d(x, z) \leq d(x, y) + d(y, z)$
\end{itemize}
\end{df}

\begin{df}
A \textbf{Cauchy sequence} is an infinite sequence that has a limit with
respect a consistent norm.
\end{df}

\begin{df}
A metric space is called \textbf{complete} if every Cauchy sequence in
$M$ has its limit in $M$.
\end{df}

\begin{df}
A \textbf{Hilbert space} is an inner product space that is also a
complete metrix space given by the norm $d(x, y) = \sqrt{\langle x - y,
x - y \rangle}$.
\end{df}

\begin{prb}
Verify that the triangle inequality holds for the above norm.
\end{prb}

\begin{proof}[Solution]
\begin{lem}[Cauchy-Schwarz]
$|\langle x, y \rangle | \leq \sqrt{\langle x, x \rangle} \sqrt{\langle y,
y \rangle}$.
\end{lem}
\begin{proof}
We have $|\langle x, y \rangle | = \sqrt{\langle x, y \rangle \langle y,
x \rangle}$, so we square to get $\langle x, y \rangle \langle y, x
\rangle \leq \langle x, x \rangle \langle y, y \rangle$, or $\langle x,
x \rangle \langle y, y \rangle - \langle x, y \rangle \langle y, x
\rangle \geq 0$. The inequality is clearly true when $x = 0$, so we
ignore that case and divide through by $\langle x, x \rangle$ to get
$\langle y, y \rangle - c \langle x, y \rangle = \langle y - cx, y
\rangle \geq 0$ where $c = \frac{\langle y, x \rangle}{\langle x, x
\rangle}$.

We can expand $\langle y - cx, y - cx \rangle = \langle y - cx, y
\rangle - \overline{c} \langle y - cx, x \rangle$, but $\langle y - cx,
x \rangle = \langle y, x \rangle - c \langle x, x \rangle = \langle y, x
\rangle - \langle y, x \rangle = 0$, so we get
\[ \langle y - cx, y \rangle = \langle y - cx, y - cx \rangle \geq 0, \]
as desired.

\end{proof}
We wish to show
\[ \sqrt{\langle x + y, x + y \rangle} \leq \sqrt{\langle x, x \rangle}
+ \sqrt{\langle y, y \rangle}. \]
Squaring gives
\[ \langle x, x \rangle + \langle x, y \rangle + \langle y, x \rangle +
\langle y, y \rangle \leq \langle x, x \rangle + \langle y, y \rangle +
2 \sqrt{\langle x, x \rangle \langle y, y \rangle}, \]
or
\[ \sqrt{\langle x, x \rangle} \sqrt{\langle y, y \rangle} +
\sqrt{\langle y, y \rangle} \sqrt{\langle x, x \rangle} \leq 2
\sqrt{\langle x, x \rangle \langle y, y \rangle} \]
by Cauchy-Schwarz, so we are done.
\end{proof}


\end{document}
