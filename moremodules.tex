\section{More Modules}
This section is reserved for some crucial propositions regarding
modules.

\subsection{Homomorphisms With Modules}

\begin{df}
We will use the notation $\hom(X, Y)$ to denote the set of all
homomorphisms from $X$ to $Y$.
\end{df}

\begin{prop}
Let $\varphi$ be a homomorphism. Then $\varphi$ is injective if and only
if $\ker \varphi = \lbrace 0 \rbrace$.
\end{prop}
\begin{proof}
We first prove that $\varphi$ is injective if $\ker \varphi = \lbrace 0
\rbrace$. Suppose $\varphi(\alpha) = \varphi(\beta)$. Then
$\varphi(\alpha) + (-\varphi(\beta)) = 0$, or $\varphi(\alpha) +
\varphi(-\beta) = 0 \iff \varphi(\alpha + (-\beta)) = 0$. However, $\ker
\varphi = \lbrace 0 \rbrace$ so $\alpha + (-\beta) = 0$, or $\alpha =
\beta$. \\
Now for the other direction. Suppose $\varphi$ is injective. We have
that $\varphi(\alpha) = \varphi(0 + \alpha) = \varphi(0) +
\varphi(\alpha) \implies \varphi(0) = 0$, so $\varphi(\alpha) = 0 =
\varphi(0) \implies \alpha = 0$ by injectivity, so we are done.
\end{proof}

\begin{prop}
Take a $R$-module $M$. The map $\chi : \hom_R(R, M) \rightarrow M$ by
$\chi(\varphi) = \varphi(1)$, or $\varphi \mapsto \varphi(1)$ is an
\textbf{isomorphism}, or a homomorphism that admits an inverse. In other
words, a homomorphism from a ring $R$ to a $R$-module is determined by
its value at $1$.
\end{prop}
\begin{proof}
Let $\varphi, \psi \in \hom_R(R, M)$ and $r \in R$. We can compute that
\[ \chi(\varphi + r \psi) = (\varphi + r \psi)(1) = \varphi(1) + r
\psi(1) = \chi(\varphi) + r \chi(\psi), \]
which gives that $\chi$ is a $R$-module homomorphism. \\
Now suppose $\varphi = \ker \chi$. Then $\forall r \in R$, 
\[ \varphi(r) = r \varphi(1) = r \chi(\varphi) = r * 0 = 0, \]
implying $\varphi$ is injective. \\
Now suppose $m \in M$. We can define a $\psi_m : R \rightarrow M$ by
$\psi_m(r) = rm$. We can easily compute that
\[ \psi_m(r_1 + k r_2) = r_1 m + (k r_2) m = r_1 m + k (r_2 m) =
\psi_m(r_1) + k \psi_m(r_2), \]
so $\psi_m$ is a $R$-module homomorphism, meaning $\psi_m \in \hom_R(R,
M)$. Moreover, $\chi(\psi_m) = \psi_m(1) = 1m = m$ so $\chi$ is
surjective. \\
Because $\chi$ is a homomorphism and is bijective, it is an isomorphism
and we are done.
\end{proof}

% \begin{prop}
% We also have the following bijections:
% \[ \hom_R(X_1 \cup X_2, Y) \leftrightarrow \hom_R(X_1, Y) \times
% \hom_R(X_2, Y) \]
% and
% \[ \hom_R(X, Y_1 \times Y_2) \leftrightarrow \hom_R(X, Y_1) \times
% \hom_R(X, Y_2) \]
% \end{prop}
% \begin{proof}
% I will include an example, taken from Evan Chen. Suppose you want to
% define a function on the set $\lbrace 1, 2, 3 \rbrace$. It is the same
% as defining a function on the set $\lbrace 1, 2 \rbrace$ and then
% another function on the set $\lbrace 3 \rbrace$. \\
% Basically, you take two projections $\alpha_1, \alpha_2$ into $X_1
% \sqcup X_2$ and the compose them with your function from $X_1 \sqcup
% X_2$ to $Y$. \\
% Seeing the logic behind the 2nd is left as an exercise to the reader.
% \end{proof}
% 
% \begin{prop}
% We can extend this to direct sums as well. Take any two $R$-modules
% $M_1, M_2$. Then there exists a bijection
% \[ \hom_R(M_1 \oplus M_2, Y) \leftrightarrow \hom_R(M_1, Y) \times
% \hom_R(M_2, Y) \]
% and
% \[ \hom_R(X, M_1 \oplus M_2) \leftrightarrow \hom_R(X, M_1) \times
% \hom_R(X, M_2) \]
% \end{prop}
% 
% \begin{rem}
% We can now consider larger direct sums $M_1, M_2, \dots, M_n$. However,
% we are more interested in the bijection
% \[ \hom_R(R^{\oplus n}, M) \leftrightarrow \hom_R(R, M)^n \simeq M^n \]
% \end{rem}
% 
% \begin{prop}
% Let $A$ be an indexing set, and consider the set of modules $(M_a)_{a
% \in A}$. Then
% \[ M = \prod_{a \in A} M_a \]
% also has a module structure.
% \end{prop}
% 
% \begin{df}
% We define the infinite direct sum $\bigoplus_a M_a$, which is a subset
% of $\prod_a M_a$, where all but finitely many indices $A$ is zero.
% \end{df}
