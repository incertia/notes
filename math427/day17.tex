\section{October 2, 2015}

\subsection{Orbits and Stabilizers}
Recall that the stabilizer of an element $x$ are the elements $g \in G$
that fix $x$ under the action.

\begin{df}
Similarly, we define the \textbf{orbit} of $G$ on an element $x$ to be
\[ Gx = \lbrace gx \mid g \in G \rbrace. \]
\end{df}

\begin{prop}
The map $\psi : G / \stab(x) \to Gx, a \stab(x) \mapsto ax$ is a well
defined bijection.
\end{prop}

\begin{proof}
$ $
\begin{enumerate}
\item Suppose $a\stab(x) = b\stab(x)$ for some $a, b \in G$. Then $a =
bh$ for some $h \in \stab(x)$. Now compute
\[ ax = (bh)x = b(hx) = bx \]
to get that $\psi$ is well defined.
\item $\psi$ is obviously surjective. $y \in Gx \implies y = gx$ and
$g \stab(x) \xmapsto{\psi} gx = y$.
\item To get that $\psi$ is injective, compute
\[ \begin{aligned}
\psi(a \stab(x)) &= \psi(b \stab(x)) \\
ax &= bx \\
b^{-1}ax &= ex = x \\
&\Downarrow \\
b^{-1}a &\in \stab(x) \\
&\Downarrow \\
a &= hb, h \in \stab(x) \\
a\stab(x) &= b\stab(x). \\
\end{aligned} \]
\end{enumerate}
\end{proof}

\begin{ex}
We introduce a very silly example.

Consider $SO_2(\RR)$ and the point $P = (r, 0)$. $SO_2(\RR)(P) = \lbrace
(x, y) \mid x^2 + y^2 = r^2 \rbrace$ and $\stab(P) = \lbrace I \rbrace$.
Then the map
\[ \begin{aligned}
SO_2(\RR) / \stab(P) = SO_2(\RR) &\longrightarrow SO_2(\RR)(P) \\
A &\longmapsto AP \\
\end{aligned} \]
is a bijection.
\end{ex}

\begin{cor}[Orbit Stabilizer Theorem]
\label{orbitstabilizer}
We have the following relationship for all $x \in X$.
\[ \lvert Gx \rvert \lvert \stab(x) \rvert = \lvert G \rvert. \]
\end{cor}

\begin{proof}
We have that $\lvert G / \stab(x) \rvert = \lvert Gx \rvert$ and
Lagrange tells us $\lvert G \rvert = \lvert G / \stab(x) \rvert \lvert
\stab(x) \rvert$.
\end{proof}

\begin{prop}
Orbits form a valid partition and equivalence relation. i.e.
\[ x \sim y \iff x = gy \quad g \in G. \]
\end{prop}

\begin{proof}
We want to show that if $Gx \cap Gy \neq \varnothing$, then $Gx = Gy$.
So suppose $z \in Gx, Gy$ so $z = g_1 x = g_2 y$. Then $x = g_1^{-1}(g_2
y) = (g_1^{-1} g_2) y \in Gy \implies gx = g((g_1^{-1}g_2)y) \implies Gx
\subseteq Gy$. Similarly, $Gy \subseteq Gx$ so $Gx = Gy$. The valid
partitioning gives us an equivalence relation.
\end{proof}

\begin{prop}
\label{actionbijection}
There is a bijection between actions of $G$ on $X$ and homomorphisms $G
\to \aut(X)$.
\end{prop}

\begin{proof}
Given an action, we get maps $\varphi_g : X \to X, x \xmapsto{\varphi_g}
gx$ which have left and right inverse $\varphi_{g^{-1}}$ so $\varphi_g
\in \aut(X)$. Group action compatability tells us that $a(bx) = (ab)x$,
or $(\varphi_a \circ \varphi_b)(x) = \varphi_a(\varphi_b(x)) =
\varphi_{ab}(x)$. Which gives that the mapping $G \to \aut(X), g \mapsto
\varphi_g$ is a homomorphism.

Conversely, given a homomorphism $\psi : G \to \aut(X)$, we can define
an action with $gx = \psi(g)(x)$. By homomorphism properties, it is
trivial to prove that this is indeed an action.
\end{proof}

\begin{rem}
There are two important actions of a group onto itself. Left
multiplication ($x \mapsto gx$) and conjugation ($x \mapsto gxg^{-1}$).
\end{rem}

\begin{df}
Orbits of conjugation are called \textbf{conjugacy classes}.
\end{df}

\begin{thm}[Cayley]
\label{cayleythm}
Left multiplication defines an injective homomorphism $\varphi : G \inj
\sym(G)$.
\end{thm}

\begin{proof}
Left multiplication is an action so we get a homomorphism by
\ref{actionbijection}.

We then compute the kernel. If $g \in \ker \varphi$, then $ge = e
\implies g = e$ so $\ker \varphi = \lbrace e \rbrace$.
\end{proof}

\begin{rem}
\ref{cayleythm} basically says that every group is isomorphic to some
subgroup of its symmetric group, or you can embed every group into its
symmetric group.
\end{rem}
