\section{September 28, 2015}

\subsection{Review}

\begin{ex}
We extend the example from yesterday and show $\ZZ / ab \ZZ \iso \ZZ / a
\ZZ \times \ZZ / b \ZZ$ for $(a, b) = 1$.
\end{ex}

\begin{proof}
Recall that $(a, b) = 1, a \mid bq \implies a \mid q$. Now we define the
homomorphism $f : \ZZ \to \ZZ / a \ZZ \times \ZZ / b \ZZ$ given by $k
\mapsto ([k]_a, [k]_b)$. We can compute $\ker(f) = ab\ZZ$. First
isomorphism theorem gives us that there is a well defined injective
homomorphism $\overline{f} : \ZZ / ab \ZZ \to \ZZ / a \ZZ \times \ZZ / b
\ZZ$ given by $[k]_{ab} \mapsto ([k]_a, [k]_b)$. Comparing the sizes of
both groups, we get that this morphism is surjective so it is also an
isomorphism between the two groups.
\end{proof}

\subsection{Unrelated Things}

We also give an application of Lagrange's Theorem.

\begin{lem}
\label{lagrange2}
Suppose $G$ is a finite group and $g \in G$. Then
\[ g^{|G|} = e. \]
\end{lem}

\begin{proof}
By Lagrange's Theorem, we have
\[ |G| = |\langle g \rangle||G/\langle g \rangle|. \]
We also have $\langle g \rangle \iso \ZZ / n \ZZ$ where $n = |\langle g
\rangle|$ and $g^n = e$. Now compute
\[ g^{|G|} = g^{|\langle g \rangle||G / \langle g \rangle|} = e^{|G /
\langle g \rangle|} = e. \]
\end{proof}

\begin{thm}[Euler's Totient Theorem]
If $(a, n) = 1$, then $a^{\varphi(n)} \equiv 1 \pmod{n}$.
\end{thm}

\begin{proof}
Let $(\ZZ / n \ZZ)^{\times}$ be the group of units of $\ZZ / n \ZZ$, aka
$\lbrace [a] \mid (a, n) = 1 \rbrace$. We have $|(\ZZ / n \ZZ)^\times| =
\varphi(n)$. Now just use \ref{lagrange2}.
\end{proof}

\subsection{Sign Homomorphism for Symmetric Groups}

\begin{df}
A permutation $\tau \in S_n$ is called a \textbf{transposition} if
$\tau = (ij)$.
\end{df}

\begin{thm}
We have the homomorphism $\sgn : S_n \to \lbrace \pm 1 \rbrace$ such
that given $\tau \in S_n$ a transposition, $\sgn(\tau) = -1$.
\end{thm}

\subsection{A Brief Note in Representation Theory}
Let $V$ be a vector space. We have a group $GL(V)$ whose elements are
invertible linear maps $T : V \to V$ under composition. In fact, we have
the isomorphism $GL_n(\RR) \iso GL(\RR^n)$. These two things are not
quite the same. The left consists of matrices and the right consists of
linear maps.

\begin{df}
A (real) \textbf{representation} of a group $G$ on a (real) vector space
$V$ is a homomorphism
\[ \varphi : G \to GL(V). \]
\end{df}

\begin{ex}
The dihedral group $D_{2n}$ has canonical representation $\varphi :
D_{2n} \to GL_2(\RR)$ where
\[ \rho \mapsto \begin{pmatrix} \cos \frac{2\pi}{n} & -\sin
\frac{2\pi}{n} \\ \sin \frac{2\pi}{n} & \cos \frac{2\pi}{n}
\end{pmatrix} \quad \tau \mapsto \begin{pmatrix} 1 & 0 \\ 0 & -1
\end{pmatrix} \]
and $\rho$ is a rotation and $\tau$ is a reflection.
\end{ex}

\begin{ex}
$S_n$ has a canonical represnetation on $\RR^n$.

We have the isomorphism
$\varphi : \RR^n \leftarrow \mathcal{F}(\lbrace 1, 2, \dots, n\rbrace)$
given by
\[ (x_1, x_2, \dots, x_n) \mapsfrom \left(\begin{aligned}
1 &\mapsto x_1 \\
2 &\mapsto x_2 \\
&\vdots \\
n &\mapsto x_n
\end{aligned}\right) \]
Notice that this bijection is linear, or
\[ \varphi(\lambda x + \mu y) = \lambda\varphi(x) + \mu\varphi(y) \]

Now notice that $S_n$ acts on $\mathcal{F}(\lbrace 1, 2, \dots,
n\rbrace)$ via $\sigma \cdot f = f \circ \sigma$. This also gives a
group action on $\RR^n$, namely
\[ \sigma \cdot (x_1, x_2, \dots, x_n) = (x_{\sigma(1)}, \dots,
x_{\sigma(n)}). \]

For each $\sigma \in S_n$, we now define the map $\psi(\sigma) : \RR^n
\to \RR^n$ by
\[ (x_1, \dots, x_n) \mapsto (x_{\sigma(1)}, \dots, x_{\sigma(n)}) \]
Notice how $\psi(\sigma)$ is linear so $\psi(\sigma) \in GL(\RR^n)$.
Compatability of group action gives us that $\psi$ is a homomorphism
from $S_n \to GL(\RR^n)$, so it is a representation.
\end{ex}
