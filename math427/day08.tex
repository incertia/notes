\section{Day 8}

\subsection{Extension of Yesterday}

Last time, we defined a subgroup generated by a generator. We now extend
this idea to a set of generators.

\begin{prop}
Let $\lbrace\mathcal{H}_\alpha\rbrace_{\alpha \in A}$ be some set of
subgroups of $G$. Then
\[ \bigcap_{\alpha \in A} \mathcal{H}_\alpha \]
is also a subgroup.
\end{prop}

\begin{proof}
This is in the homework.
\end{proof}

\begin{prop}
Let $G$ be a group and let $X \subseteq G$ be a subset of the elements
of $G$. Let $Y$ be the set of subgroups $\mathcal{H}$ such that
$X \subseteq \mathcal{H}$. Then
\[ \bigcap_{\mathcal{H} \in Y} \mathcal{H} \]
is a subgroup containing $X$. In fact, it is the smallest subgroup
containing $X$.
\end{prop}

\begin{proof}
Suppose $K$ is a subgroup containing $X$, or $K \in Y$. Then
$\bigcap_{\mathcal{H} \in Y} \mathcal{H} \subseteq K$. Now just take $K$
to be the smallest subgroup containing $X$.
\end{proof}

\begin{df}
Define the \textbf{dihedral group} of size $2n$ to be $D_{2n}$, which is
composed of the $n$ reflections and $n$ rotations of the regular
$n$-gon. If one reflection is characterized as $T$ and one rotation is
characterized as $R$, then
\[ D_{2n} = \lbrace 1, R, R^2, R^3, \dots, R^n, T, TR, TR^2, \dots, TR^n
\rbrace. \]
\end{df}

\subsection{Other Things?}
Quiz today. Also lots of examples I don't want to write down.
