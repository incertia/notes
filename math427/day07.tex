\section{Day 7 - New Things}

\begin{df}
A non-empty subset $H$ of a group $G$ is called a \textbf{subgroup} of
$G$ if $H$ is a group with respect to the binary operation of $G$.
\end{df}

\begin{ex}
$\ZZ$ is a subgroup of $\RR$ under $+$.
\end{ex}

\begin{rem}
If $H$ is a subgroup of $G$, then $e \in H$.
\end{rem}

\begin{rem}
The inclusion map $i : H \to G, i : h \mapsto h$ is a homomorphism.
\end{rem}

\begin{prop}
\label{subgroupcriteria}
Let $H \subseteq G$ and $H \neq \varnothing$. Then $H$ is a subgroup iff
$\forall h_1, h_2 \in H$, $h_1 h_2^{-1} \in H$.
\end{prop}

\begin{proof}
Take an $h \in H$. $e = h h^{-1} \in H$. For all $h \in H$, we have
$h^{-1} = e h^{-1} \in H$. Additionally, for all $h \in H$, $h^{-1} \in
H$ so $h_1 (h_2^{-1})^{-1} = h_1 h_2 \in H$.

The reverse direction is trivial by the definition of subgroup.
\end{proof}

\begin{prop}
Suppose $H_1, H_2$ are subgroups of $G$. Then $H_1 \cap H_2$ is also a
subgroup.
\end{prop}

\begin{proof}
By \ref{subgroupcriteria}, we just need $ab^{-1} \in H_1 \cap H_2$. So
take $a, b \in H_1 \cap H_2$ and write $ab^{-1} \in H_1, ab^{-1} \in
H_2$ because $H_1, H_2$ are subgroups, which implies $ab^{-1} \in H_1
\cap H_2$.
\end{proof}

\begin{thm}
Let $H_{\alpha \in A}$ be a collection of subgroups of $G$. Then
\[ \bigcap_{\alpha \in A} H_\alpha \]
is a subgroup of $G$.
\end{thm}

\begin{proof}
This is a homework problem.
\end{proof}

\begin{df}
The \textbf{image} of a map $f : A \to B$ is the set denoted by $f(A) =
\lbrace f(a) : a \in A \rbrace$.
\end{df}

\begin{prop}
Suppose $\varphi : G \to H$ is a homomorphism. Then $\varphi(G)$ is a
subgroup of $H$.
\end{prop}

\begin{proof}
Trivial.
\end{proof}

\begin{df}
Define the \textbf{cyclic group} generated by an element $g \in G$ to be
the subgroup $\lbrace g^n : n \in \ZZ \rbrace$. Denote this by $\langle
g \rangle$.
\end{df}

\begin{df}
The \textbf{order} of an element $g \in G$ is the size of $\langle g
\rangle$.
\end{df}

\begin{rem}
For any $\langle g \rangle \subseteq G$, $\langle g \rangle \cong
\ZZ/n\ZZ$ if $\langle g \rangle$ is finite.
\end{rem}
