\section{September 16, 2015}

\subsection{``Unique Factorization'' of Symmetric Group}
We spent a lot of time yesterday laying the groundwork for a very
important ``factorization'' theorem regarding permutations.

\begin{thm}
Every permutation $\sigma \in S_n$ can be written as a unique product of
disjoint cycles up to order.

We can restate in the following way. Let $X$ be a set of finite
elements. Then any $\sigma \in \aut(X)$ can be written as a unique
product of disjoint cycles.
\end{thm}

\begin{proof}
We first prove the existence of such a product by induction on $|X|$. If
$|X| = 1$, then $\aut(X) = \lbrace \id_X \rbrace$. Now suppose that it
is true for all $k \leq n$. Let $\sigma \in X, \sigma \neq \id, |X| = n
+ 1$.  Because $\sigma \neq \id$, we have some $x$ such that $\sigma(x)
\neq x$ and we can create a chain $x_0 = x$, $x_i = \sigma(x_{i - 1})$
for $i > 0$. Clearly not all of these are distinct because $X$ is
finite.

Now we take the largest $k$ such that $x_0, x_1, \dots, x_k$ are
distinct. We claim that $\sigma(x_k) = x_0$.

Suppose $\sigma(x_k) = x_i, i > 0$. Then $\sigma(x_k) = \sigma(x_{i -
1})$ and $\sigma$ being a bijection implies that $x_k = x_{i - 1}$,
which is a contradiction because $x_0, x_1, \dots, x_k$ are all
distinct.

We now define the sets $X_1 = \lbrace x_0, x_1, \dots, x_k \rbrace, Y =
X \setminus X_1$. By construction, $\sigma(X_1) = X_1$ so $\sigma(Y) =
Y$. Now let
\[ \sigma_1 = \left\lbrace\begin{aligned}
\sigma(x) &\quad x \in X \\
x &\quad x \in Y \\
\end{aligned}\right. \quad \sigma_2 = \left\lbrace\begin{aligned}
x &\quad x \in X \\
\sigma(x) &\quad x \in Y \\
\end{aligned}\right. \]
This gives $\sigma_1, \sigma_2$ as disjoint permutations and $\sigma_2$
is a product of disjoint permutations by inductive assumption so we are
done.

For uniqueness, we again induct on $|X|$. For $|X| = 1$, the result is
obviously true. Now let $X$ be a set such that $|X| = n + 1$. Pick a
nontrivial permutation $\sigma \in \aut(X)$ and write $\sigma = \sigma_1
\circ \sigma_2 \circ \cdots \circ \sigma_r = \tau_1 \circ \tau_2 \circ
\cdots \circ \tau_s$ where the $\sigma_i$ are pairwise distinct and the
$\tau_i$ are pairwise distinct.

Pick some $\sigma_i$, we will get that $\sigma_i = \tau_j$ for some $j$.
This is because we have some $x$ such that $\sigma_{j \neq i}(x) \neq y$
and $\sigma_i(x) = y$. There is some $\tau_j$ such that $\tau_j(x) \neq
x$. Suppose $\tau_{k \neq j}(z) = y$. Then $\tau(x) = \tau(z) = y$ but
$x$ and $z$ are in two different cycles of $\tau$ which is a
contradiction, so $\tau_j(x) = y$. We can now ``factor'' out $\sigma_i$
and $\tau_j$ and apply the induction hypothesis.
\end{proof}

\subsection{Group Action}

\begin{df}
Let $X$ be a set. We say a group $G$ \textbf{acts} on $X$ via an
\textbf{action} $\alpha$ if there exists an $\alpha : G \times X \to X$
notated $g \cdot x = \alpha(g, x)$ such that
\begin{enumerate}
\item $e \cdot x = x, \forall x \in X$.
\item $g \cdot (h \cdot x) = (gh) \cdot x, \forall x \in X$.
\end{enumerate}
\end{df}

\begin{ex}
$\aut(X)$ acts on $X$ via function application.
\end{ex}

\begin{df}
Let the set of all real valued functions on a set $X$ be notated as
$\mathcal{F}(X) = \lbrace f : X \to \RR \rbrace$.
\end{df}

\begin{ex}
If $G$ acts on $X$, then $G$ acts on $\mathcal{F}(X)$ via $(gf)(x) =
f(g^{-1}x)$.
\end{ex}

\begin{proof}
We clearly have $(ef)(x) = f(x)$. Now compute
\[ \begin{aligned}
((gh)f)(x) &= f((gh)^{-1}x) \\
&= f((h^{-1}g^{-1})x) \\
&= f(h^{-1}(g^{-1}x) \\
&= (hf)(g^{-1}x) \\
&= (g(hf))(x) \\
\end{aligned} \]
\end{proof}
