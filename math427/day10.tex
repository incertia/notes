\section{Day 10}

\subsection{More Permutations}
We spent a lot of time yesterday laying the groundwork for a very
important ``factorization'' theorem regarding permutations.

\begin{thm}
Every permutation $\sigma \in S_n$ can be written as a unique product of
disjoint cycles up to order.

We can restate in the following way. Let $X$ be a set of finite
elements. Then any $\sigma \in \aut(X)$ can be written as a unique
product of disjoint cycles.
\end{thm}

\begin{proof}
We first prove the existence of such a product by induction on $|X|$. If
$|X| = 1$, then $\aut(X) = \lbrace \id_X \rbrace$. Now suppose that it
is true for all $k \leq n$. Let $\sigma \in X, \sigma \neq \id, |X| = n
+ 1$.  Because $\sigma \neq \id$, we have some $x$ such that $\sigma(x)
\neq x$ and we can create a chain $x_0 = x$, $x_i = \sigma(x_{i - 1})$
for $i > 0$. Clearly not all of these are distinct because $X$ is
finite.

Now we take the largest $k$ such that $x_0, x_1, \dots, x_k$ are
distinct. We claim that $\sigma(x_k) = x_0$.

Suppose $\sigma(x_k) = x_i, i > 0$. Then $\sigma(x_k) = \sigma(x_{i -
1})$ and $\sigma$ being a bijection implies that $x_k = x_{i - 1}$,
which is a contradiction because $x_0, x_1, \dots, x_k$ are all
distinct.

We now define the sets $X_1 = \lbrace x_0, x_1, \dots, x_k \rbrace, Y =
X \setminus X_1$. By construction, $\sigma(X_1) = X_1$ so $\sigma(Y) =
Y$. Now let
\[ \sigma_1 = \left\lbrace\begin{aligned}
\sigma(x) &\quad x \in X \\
x &\quad x \in Y \\
\end{aligned}\right. \quad \sigma_2 = \left\lbrace\begin{aligned}
x &\quad x \in X \\
\sigma(x) &\quad x \in Y \\
\end{aligned}\right. \]
This gives $\sigma_1, \sigma_2$ as disjoint permutations and $\sigma_2$
is a product of disjoint permutations by inductive assumption so we are
done.
\end{proof}
