\section{Day 6 - More Examples}

\begin{df}
A group is said to be commutative or \textbf{abelian} if the operation
commutes.
\end{df}

\begin{ex}
Let $X$ be a set. Then the set of automorphisms $\aut(X) = \lbrace f : X
\to X \mid f \textrm{ invertible} \rbrace$ is a group under composition.
\end{ex}

\begin{prop}
$(a^{-1})^{-1} = a$.
\end{prop}

\begin{proof}
$a^{-1} * a = e$.
\end{proof}

\begin{df}
Let $G, H$ be two groups. A map $\varphi : G \to H$ is called a
\textbf{group homomorphism} iff $\varphi(a * b) = \varphi(a) *
\varphi(b) \quad \forall a, b \in G$.
\end{df}

\begin{ex}
We have the homomorphism from $\RR$ under $+$ to $\RR^*$ under $\cdot$
by $\exp(x) = e^x$.
\end{ex}

\begin{ex}
$\det : GL_n(F) \to F^*$ is a homomorphism.
\end{ex}

\begin{df}
Define the identity map on a set $X$ to be $\id_X : X \to X$ given by $x
\mapsto x$.
\end{df}

\begin{df}
A homomorphism $\varphi : G \to H$ is an \textbf{isomorphism} if there
is an inverse $\psi : H \to G$ such that $\varphi \circ \psi = \id_G,
\psi \circ \varphi = \id_H$.
\end{df}

\begin{prop}
Let $\varphi : G \to H$ be a bijective homomorphism. Then $\varphi^{-1}
: H \to G$ is a homomorphism.
\end{prop}

\begin{proof}
We want $\varphi^{-1}(xy) = \varphi^{-1}(x) \varphi^{-1}(y)$. Take
\[ \varphi(\varphi^{-1}(x) \varphi^{-1}(y)) = \varphi(\varphi^{-1}(x))
\varphi(\varphi^{-1}(y)) = xy = \varphi(\varphi^{-1}(xy)). \]
Injectivity on $\varphi$ gives the desired result.
\end{proof}

\begin{ex}
$\ln : (0, \infty) \to \RR^+$ has the inverse $\exp : \RR \to (0,
\infty)$.
\end{ex}
