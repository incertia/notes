\section{September 18, 2015}

\subsection{More Subgroups}

\begin{prop}
\label{homomorphismidea}
Let $G$ be a group and let $G \ni g \neq e$. Let $\varphi : \ZZ \to
\langle g \rangle$ be given by $k \mapsto g^k$. Then either
\begin{enumerate}
\item $\varphi$ is an isomorphism.
\item $\exists n \geq 0 \st \overline{\varphi} :  \ZZ / n \ZZ \to
\langle g \rangle, \overline{\varphi}([k]) = g^k$ is a well defined
isomorphism.
\end{enumerate}
\end{prop}

\begin{proof}
If $\varphi$ is injective, then $\varphi$ is an isomorphism so suppose
$\varphi$ is not injective. Then we have some $k > l$ such that
$\varphi(k) = \varphi(l)$, or $g^{k - l} = e$, or $\exists m \st g^m =
e$. Let $n$ be the minimum of these numbers.

All that remains to check is that $\overline{\varphi}$ is a well defined
homomorphism, but this is obvious.
\begin{enumerate}
\item $[k] = [l] \implies k = l + tn \implies \overline{\varphi}([k]) =
g^{l + tn} = g^l e^t = g^l = \overline{\varphi}([l])$.
\item $\overline{\varphi}([k] + [l]) = \overline{\varphi}([k + l]) =
g^{k + l} = g^k g^l = \overline{\varphi}([k]) \overline{\varphi}([j])$.
\item $\overline{\varphi}$ is clearly surjective because for each $g^k
\in \cycgroup{g}$, we have $[k] \mapsto g^k$. Now suppose
$\overline{\varphi}([a]) = \overline{\varphi}([b])$. This implies $g^a =
g^b$, or $g^{a - b} = e$, or $n \mid a - b$, so $[a] = [b]$.
\end{enumerate}
\end{proof}

\begin{rem}
We can imagine this situation with the following commutative diagram.
\begin{center}
\begin{tikzpicture}[scale=2.5, nodes={scale=1.2}]
\node (Z) at (0, 1) {$\ZZ$};
\node (g) at (1, 1) {$\cycgroup{g}$};
\node (znz) at (0, 0) {$\ZZ / n \ZZ$};
\path[->]
(Z) edge node[above]{$\varphi$} (g)
(Z) edge node[left]{$\pi$} (znz);
\path[->, dashed]
(znz) edge node[below, sloped]{$\overline{\varphi}$} (g);
\end{tikzpicture}
\end{center}
\end{rem}

We now generalize \ref{homomorphismidea} to homomorphisms in general.

\begin{df}
Let $\varphi : G \to H$ be a homomorphism. Let the \textbf{kernel} of
$\varphi$, denoted $\ker(\varphi)$ to be the set $\lbrace g \in G :
\varphi(g) = e_H \rbrace$.
\end{df}

\begin{prop}[Kernel Injectivity Relation]
Let $\varphi : G \to H$ be a homomorphism. Then
\[ \forall a, b \in G \quad \varphi(a) = \varphi(b) \iff ab^{-1} \in
\ker(\varphi). \]
More specifically, $\ker(\varphi) = \lbrace e_G \rbrace \iff \varphi$ is
injective.
\end{prop}

\begin{proof}
\[ \begin{aligned}
ab^{-1} \in \ker(\varphi) &\iff e_H = \varphi(ab^{-1}) = \varphi(a)
\varphi(b^{-1}) = \varphi(a) \varphi(b)^{-1} \\ &\iff e_h \varphi(b) =
\varphi(a) \varphi(b)^{-1} \varphi(b) \\
&\iff \varphi(b) = \varphi(a) \\
\end{aligned} \]

Now suppose $\ker(\varphi) = \lbrace e_G \rbrace$ and $\varphi(a) =
\varphi(b)$. Then $ab^{-1} \in \ker(\varphi) \implies ab^{-1} = e_G
\implies a = b$. Conversely suppose $\varphi$ is injective and $a \in
\ker(\varphi)$. Then $\varphi(a) = e_H = \varphi(e_G)$. Injectivity of
$\varphi$ gives $a = e_G$.
\end{proof}

\begin{df}
A subgroup $H \leq G$ is called \textbf{normal} if $\forall g \in G, h
\in H$, $ghg^{-1} \in H$.
\end{df}

\begin{prop}
Let $\varphi : G \to H$ be a homomorphism. Then $\ker \varphi$ is a
normal subgroup of $G$.
\end{prop}

\begin{proof}
$ $
\begin{enumerate}
\item $g_1, g_2 \in \ker \varphi \implies g_1 g_2 \in \ker \varphi$ because
$\varphi(g_1 g_2) = \varphi(g_1) \varphi(g_2) = e_H e_H = e_H$.
\item $e_G \in \ker \varphi$ because $e_H = \varphi(g) = \varphi(e_G g)
= \varphi(e_G) \varphi(g) = \varphi(e_G), \forall g \in \ker \varphi$.
\item $g \in \ker \varphi \implies g^{-1} \in \ker \varphi$ because $e_H
= \varphi(e_G) = \varphi(g g^{-1}) = \varphi(g) \varphi(g^{-1})$.
\item $ghg^{-1} \in \ker \varphi$ because $\varphi(ghg^{-1}) =
\varphi(g) \varphi(h) \varphi(g^{-1}) = \varphi(g) \varphi(g^{-1}) =
\varphi(gg^{-1}) = e_H$.
\end{enumerate}
\end{proof}

\begin{rem}
If $G$ is abelian, then any subgroup $H \leq G$ is normal.
\end{rem}

\begin{ex}
$\det : GL_n(\RR) \to \RR^\times$ is a homomorphism. We also have
$SL_2(\RR) = \ker(\det) = \lbrace A \in GL_n(\RR) : \det A = 1 \rbrace$
is a normal subgroup.
\end{ex}
