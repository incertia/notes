\section{October 23, 2015}

\subsection{Review}

\begin{rem}
We have some interesting trivial corollaries from ideals.
\begin{enumerate}
\item $1 \in I \implies I = R$.
\item $u \in R^\times, u \in I \implies I = R$.
\item If $F$ is a field, the only ideals are $0, F$.
\end{enumerate}
\end{rem}

\subsection{Quotient Rings and The First Isomorphism Theorem (again)}
\begin{thm}
Let $I$ be an ideal of some ring $R$. Then
\begin{enumerate}
\item The \textbf{quotient group} $R / I = \lbrace a + I \mid a \in R
\rbrace$ is a ring with multiplication given by $(a + I)(b + I) = ab +
I$.
\item $\pi : R \to R / I$ is a ring homomorphism.
\item $1 \mapsto 1 + I = 1_{R / I}$.
\end{enumerate}
$R / I$ is now known as the \textbf{quotient ring} of $R$ mod $I$.
\end{thm}

\begin{proof}
Trivial.
\end{proof}

\begin{prop}
If $\varphi : R \to R'$ is a ring homomorphism, then $\varphi(R)$ is a
subring of $R'$.
\end{prop}

\begin{thm}[First Isomorphism Theorem]
Let $\varphi : R \to R'$ be a ring homomorphism. Then there exists a
unique ring homomorphism $\overline{\varphi} : R / \ker \varphi \to R'$
given by $r + I \xmapsto{\overline{\varphi}} \varphi(r)$.
\end{thm}

\begin{proof}
We easiliy get that $R / \ker\varphi \to R'$ is a well defined group
homomorphism, so we only need to check it for multiplication. But this
is easy because
\[ \overline{\varphi}((a + I)(b + I)) = \overline{\varphi}(ab + I) =
\varphi(ab) = \varphi(a)\varphi(b) = \overline{\varphi}(a + I)
\overline{\varphi}(b + I). \]
\end{proof}

\begin{ex}
Canonical inclusion $\iota : \RR \to \CC$ is a ring homomorphism, so we
get some $f : \RR[X] \to \CC$ with $X \mapsto \sqrt{-1}$. In particular,
\[ \sum_{j = 0}^n a_j X^j = \sum_{j = 0}^n a_j (\sqrt{-1})^j. \]
Trivially, $\ker f = (X^2 + 1)\RR[X]$.
\end{ex}

\subsection{Generating Ideals}
Similarly to generating subgroups from a set of elements, we can also
generate ideals. Namely,
\[ \bigcap_{I \in A} I = \lbrace i \mid i \in I \quad \forall I \in A
\rbrace \]
for some set of ideals $A$ is also an ideal. Now let $A$ be the set of
ideals all containing some ideal $S$. Then $\cycgroup{S}$ is the
smallest ideal containing $S$.

\begin{df}
$\cycgroup{\lbrace a \rbrace} = \lbrace ar \mid r \in R \rbrace = aR$ is
called the \textbf{principal ideal} generated by $a$.
\end{df}
