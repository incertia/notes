\section{Day 2 - GCD}

\begin{df}
An integer $a \in \mathbb{Z}$ \textbf{divides} $b \in \mathbb{Z}$ if
$\exists q \in \mathbb{Z} : b = qa$, notated by $a \mid b$. If not, then
$a \nmid b$.
\end{df}

\begin{df}
The \textbf{greatest common divisor}, or \textbf{gcd} of $a, b \in
\mathbb{Z}$, denoted $\gcd(a, b) = (a, b)$, is a $d \in \mathbb{Z}_0^+$
such that
\[ \begin{aligned}
d \mid a&, d \mid b \\
c \mid a&, c \mid b \implies c \mid d.
\end{aligned} \]
\end{df}

\begin{thm}[Bezout]
Let $a, b \in \mathbb{Z}$ such that they are not both zero. Then
$\gcd(a, b)$ exists and $\exists x, y \in \mathbb{Z} : (a, b) = xa +
yb$.
\end{thm}

\begin{proof}
Let $S = \lbrace ua + vb : u, v \in \mathbb{Z}, ua + vb > 0 \rbrace$.
One of $aa, bb \in S$ so well ordering gives us a $d = \min S$. We claim
$d = \gcd(a, b)$.

By division, we have $q, r \in \mathbb{Z} : a = qd + r, 0 \leq r < d$.
If $r \neq 0$, then $d > r = a - qd \implies r = ua + vb \implies r \in
S$, contradicting the minimality of $d$.

If $c \mid a, c \mid b$, then $c \mid xa + yb = d$.
\end{proof}
