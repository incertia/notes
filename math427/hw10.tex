\documentclass{article}

\usepackage{incertia}

% setup the header
\pagestyle{fancy}
\lhead{Will Song}
\chead{Math 427}
\rhead{HW10}

\begin{document}

\begin{enumerate}

\item
Suppose for the sake of contradiction that $n$ is not prime.
$x \in \cycgroup{1_R} \implies x \in R$ which gives us that there is
some $a, b \in \cycgroup{1_R} \subseteq R, \varphi : \cycgroup{1_R} \to
\ZZ / n \ZZ, a, b \neq 0$ such that $\varphi(ab) = 0$, but $R$ is an
integral domain so zero divisors like this is impossible.

\item
\begin{enumerate}[(a)]
\item
Take some $n \in N, r \in R, n^k = 0$ and compute $(nr)^k = n^k r^k$ by
commutativity.

\item
If $R / N$ has a non-zero nilpotent element $aN$, then $(aN)^k = (a^k)N
= 0N \implies aN = 0N$.

\item
Suppose $n \in N$ with $n^k = 0, n^{k - 1} \neq 0$. Then $0 = \varphi(0)
= \varphi(n^k) = \varphi(n) \varphi(n^{k - 1})$ so at least one of
$\varphi(n), \varphi(n^{k - 1}) = 0$ but any one of these gives that the
other is $0$ due to $(k - 1, k) = 1$. Basically if $\varphi(n) = 0$,
then $\varphi(n)^{k - 1} = \varphi(n^{k - 1}) = 0$ and $\varphi(n^{k -
1}) = 0$ gives us $\varphi(n) = 0$ by $\left(n^{k - 1}\right)^{k - 1} =
n$. This implies that $n \in \ker \varphi$ and $N \subseteq \ker
\varphi$
\end{enumerate}

\item
Rewrite as $a^2 = 1a$. Suppose $a \neq 0$. From the cancellation law of
integral domains, we get $a = 1$. $a = 0$ also clearly solves this
equation.

\item
We just expand a product of two polynomials $p_1 p_2$, with $p_1 = a_0 +
a_1 X + \cdots + a_p x^p, p_2 = b_0 + b_1 X + \cdots + b_q X^q$ and
induct on its degree. The base case of degree $1$ is obviously true. If
$p_1 p_2 = 0$ then it is necessary that $a_0 b_0 = 0$, so one of those
must be zero and we can reduce the degree of one of $p_1, p_2$. By the
induction hypothesis one of these polynomials is zero so we are done.

\item
We need to show that some sort of Euclidean division exists in
$\ZZ[\sqrt{-2}]$ with norm $d(a + b \sqrt{-2}) = a^2 + 2b^2$, or for any
choice of $n_1, n_2, d_1, d_2$, there exists $q_1, q_2, r_1, r_2$ such
that
\[ n_1 + n_2 \sqrt{-2} = (q_1 + q_2 \sqrt{-2})(d_1 + d_2 \sqrt{-2}) +
(r_1 + r_2 \sqrt{-2}) \]
with 
\[ 0 \leq r_1^2 + 2r_2^2 \leq d_1^2 + 2d_2^2. \]
We can expand the required division to
\[ n_1 + n_2 \sqrt{-2} = (q_1 d_1 - 2q_2 d_2 + r_1) + (q_1 d_2 + d_1 q_2
+ r_2)\sqrt{-2}. \]

\item $\ZZ[i]$ is a PID, so every ideal is generated by a single
element. Namely, $\cycgroup{2 + i} = \lbrace (2 + i)(a + bi) = (2a - b)
+ (a + 2b)i \mid a, b \in \ZZ
\rbrace$.

\end{enumerate}

\end{document}
