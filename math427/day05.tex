\section{Day 5}

\subsection{Review}
Last time, we proved that equivalence relation $\Leftrightarrow$
partitions. However, we did not prove that the two implicit maps used in
the proof of \ref{eqpart} are inverses of each other.

We also got that the operations
\[ \begin{aligned}
+ : \ZZ/n\ZZ \times \ZZ/n\ZZ &\to \ZZ/n\ZZ \\
([a], [b]) &\mapsto [a + b] \\
* : \ZZ/n\ZZ \times \ZZ/n\ZZ &\to \ZZ/n\ZZ \\
([a], [b]) &\mapsto [a \cdot b] \\
\end{aligned} \]
are well defined.

\subsection{Groups}
\begin{df}
\label{binop}
A \textbf{binary operation} $\star$ on a set $S$ is a map $\star : S
\times S \to S$ which is often notated $a \star b := \star(a, b)$.
\end{df}

\begin{df}
\label{assoc}
A binary operation $\star$ is \textbf{associative} if $a \star (b \star
c) = (a \star b) \star c$.
\end{df}

\begin{df}
\label{commute}
A binary operation $\star$ is \textbf{commutative} if $a \star b = b
\star a$.
\end{df}

\begin{df}
\label{group}
A \textbf{group} is a set $G$ together with a binary operation $* : G
\times G \to G$, usually notated $(G, *)$, and a distinguished element
$e \in G$ such that
\begin{enumerate}
\item $*$ is associative.
\item $e * a = a = a * e \quad \forall a \in G$.
\item $\forall a \in g, \exists b \in G \st a * b = e = b * a$.
\end{enumerate}
\end{df}

\begin{prop}
In any group, inverses are unique.
\end{prop}

\begin{proof}
Suppose $b, c$ are inverses of $a$. Compute
\[ b = e * b = (c * a) * b = c * (a * b) = c * e = c. \]
\end{proof}

\begin{cor}
We can define a function $\operatorname{inv} : G \to G$ such that $g *
\operatorname{inv}(g) = e$. This is normally denoted $-a$ when the
operation is $+$ and $a^{-1}$ otherwise.
\end{cor}

\begin{df}
Define $GL_n(F) := \lbrace A \in M_n(F) : \det(A) \neq 0 \rbrace$ as the
set of invertible $n \times n$ matrices over $F$.
\end{df}

\begin{prop}
$GL_n(\RR)$ is a group with the usual multiplication.
\end{prop}
