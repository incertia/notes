\section{September 30, 2015}

\subsection{Sign}

We mentioned the sign of a permutation in the symmetric group last time,
now we show that it is actually a homomorphism.

We construct the function
\[ \Delta : \RR^n \to \RR, (x_1, x_2, \dots, x_n) \mapsto \prod_{i < j}
(x_i - x_j). \]

We take a permutation and act on $\Delta$, so we have
\[ (\sigma \cdot \Delta)(x_1, \dots, x_n) = \Delta(x_{\sigma(1)}, \dots,
x_{\sigma(n)}) = \pm \Delta(x_1, \dots, x_n). \]

We also get that
\[ (ij) \cdot \Delta = -\Delta. \]

\begin{df}
Define $\sgn : S_n \to \lbrace \pm 1 \rbrace$ by
\[ \sigma \mapsto \frac{\sigma \cdot \Delta}{\Delta}. \]
\end{df}

\begin{prop}
$\sgn$ is a homomorphism.
\end{prop}

\begin{proof}
We compute
\[ \sgn(\sigma \tau) = \frac{(\sigma \tau) \cdot \Delta}{\Delta} =
\frac{\sigma \cdot (\tau \cdot \Delta)}{\Delta} = \frac{\sigma \cdot
(\sgn(\tau) \Delta)}{\Delta} =  \sgn(\tau)\frac{\sigma \cdot
\Delta}{\Delta} = \sgn(\tau) \sgn(\sigma) = \sgn(\sigma) \sgn(\tau). \]
\end{proof}

\begin{prop}
Any transposition $(ij)$ is a product of an odd number of transpositions
of the form $(k, k + 1)$.
\end{prop}

\begin{proof}
We move $i$ to $j$ with $j - i$ transpositions, then we move $j$ from $j
- 1$ to $i$ with $j - i - 1$ transpositions, so there are $2(i - j) + 1$
transpositions here.
\end{proof}

\begin{df}
We define the \textbf{alternating group} of $n$ elements to b $A_n =
\ker(\sgn)$.
\end{df}

\begin{ex}
We can define the determinant of a matrix by
\[ \det((a_{i,j})) = \sum_{\sigma \in S_n} \sgn(\sigma) \prod_{i}
a_{i,\sigma(i)} \]
\end{ex}

\begin{prop}
Any cycle in $S_n$ is a product of transpositions.

Consequently, we have $S_n$ generated by
\[ \lbrace (12), (23), \dots, (n - 1\,n) \rbrace. \]
\end{prop}

\begin{proof}
$(a_1 a_2 \cdots a_n) = (a_1 a_2)(a_2 a_3) \cdots (a_{n - 1}a_n)$.
\end{proof}

\subsection{Beginning Orbits and Stabilizers}
\begin{df}
Let $G$ act on $X$. Then the \textbf{stabilizer} of $x$ is the set
\[ \stab(x) = \lbrace g \in G : gx = x \rbrace. \]
\end{df}

\begin{prop}
$\stab(x) \leq G$.
\end{prop}

\begin{proof}
Trivial.
\end{proof}
