\section{October 9, 2015}

\subsection{Conjucacy}
Recall that there is a canonical bijection between $G / \orb(x) \biject
Gx$.

\begin{df}
A point $x \in X$ is called a \textbf{fixed point} of the action $G
\times X \to X$ if $\orb(x) = G$.
\end{df}

\begin{ex}
$(x, x, x, x, \dots, x) \in \RR^n$ is a fixed point under the standard
action given by $S_n$.
\end{ex}

\begin{ex}
There are no fixed points on a group $G$ by left multiplication unless
$\lvert G \rvert = 1$.
\end{ex}

\begin{df}
The set of all fixed points by conjugation is called the \textbf{center}
of $G$. This is denoted by
\[ Z(G) = \lbrace g \in G \mid zg = gz, \forall z \in G \rbrace. \]
\end{df}

\begin{rem}
$Z(G)$ is an abelian subgroup and hence it is normal.
\end{rem}

\begin{df}
Let $G$ act on itself by conjugation. We call the stabilizers of $x$ via
this action the \textbf{centralizer} of $x$, denoted
\[ \cent_G(x) = \stab(x) = \lbrace g \in G \mid gxg^{-1} = x \rbrace. \]
We can extend this to a subset $S \subseteq G$ by
\[ \cent_G(S) = \lbrace g \in G \mid gsg^{-1} = s, \forall s \in S
\rbrace. \]
\end{df}

\begin{df}
Similarly, we define the \textbf{normalizer} of a subset $S \subseteq
G$ to be
\[ \normal_G(S) = \lbrace g \in G \mid gsg^{-1} \in S, \forall s \in S
\rbrace. \]
\end{df}

\begin{rem}
$x \in Z(\cent_G(x))$.
\end{rem}

\begin{prop}
Suppose $G$ is a finite group and $\lvert G \rvert = p^k$ for some prime
$p$. Then $p \mid \lvert Z(G) \rvert$. In particular, $\lvert Z(G)
\rvert \geq p$.
\end{prop}

\begin{proof}
The proof requires the so called \textbf{conjugacy class equation}.
\begin{lem}[Class Equation]
Let $G$ be a finite group. Then choose representative elements $x_i$
from each orbit of $G$ (there are finitely many!) and we have
\[ \lvert G \rvert = \lvert Z(G) \rvert + \sum_{i} \frac{\lvert G
\rvert}{\lvert \cent_G(x_i) \rvert}. \]
\end{lem}
\begin{proof}
$G$ is finite so it has finitely many orbits under conjugation. In
particular, there are finitely many orbits $\mathcal{O}_1, \dots,
\mathcal{O}_n$ with size $> 1$. The union of the singleton orbits is
just $Z(G)$. Now we choose representative elements from the
non-singleton orbits $x_i \in \mathcal{O}_i, \forall i \in [1..n]$ and
we trivially have that
\[ \lvert G \rvert = \lvert Z(G) \rvert + \sum_i \lvert Gx_i \rvert. \]
Now recall that $\lvert Gx_i \rvert = \lvert G / \stab(x_i) \rvert =
\lvert G / \cent_G(x_i) \rvert = \frac{\lvert G \rvert}{\lvert
\cent_G(x_i) \rvert}$ where the equalities come from the canonical
bijection from orbit to stabilizer coset group, definition of
centralizer, and Lagrange's theorem. This yields the desired result.
\end{proof}

Now by the class equation, we have that
\[ p^k = \lvert Z(G) \rvert + \sum_i \frac{p^k}{\lvert \cent_G(x_i)
\rvert|}. \]
We also have $\lvert G / \cent_G(x_i) \rvert > 1 \quad \forall i$ so $p
\mid \frac{p^k}{\lvert \cent_G(x_i) \rvert} \quad \forall i$ from
Lagrange. This implies that $p$ divides the difference, or $p \mid
\lvert Z(G) \rvert$.
\end{proof}

\begin{prop}
This is also more of an example, but whatever. Suppose $\lvert G \rvert
= p^2$ for some prime $p$. Then either $G \iso \ZZ / p^2 \ZZ$ or $G \iso
\ZZ / p \ZZ \times \ZZ / p \ZZ$. In particular, $G$ is abelian.
\end{prop}

\begin{proof}
Take some $g \in G, g \neq e$ and look at $\cycgroup{g}$. Either $\lvert
\cycgroup{g} \rvert = p$ or $\lvert \cycgroup{g} \rvert = p^2$. If
$\lvert \cycgroup{g} \rvert = p^2$, then $G = \cycgroup{g} \iso \ZZ /
p^2 \ZZ$.

Else, suppose there is no $g \in G, g \neq e$ such that $\cycgroup{g} =
G$. We then get $\lvert \cycgroup{g} \rvert = p$ for all $g \in G$. We
also have that $Z(G) \geq p$, so pick some $g \in Z(G)$ and $h \in G
\setminus \cycgroup{g}$. Then $\cycgroup{h} \cap \cycgroup{g} = \lbrace
e \rbrace$. We also have that
\[ g^k h^l = h^l g^k \quad \forall k, l \in \ZZ \]
because $g \in Z(G) \implies \cycgroup{g} \subseteq Z(G)$. Now consider
the map $f : \cycgroup{h} \times \cycgroup{g} \to G$ given by $(h^k,
g^l) \xmapsto{f} h^k g^l$. This is trivially a homomorphism by
commutative properties of $\cycgroup{g}$. This is also trivially
injective and by comparing cardinalities of the left and right, we get
that this $f$ is bijective so it is an isomorphism. $\cycgroup{h},
\cycgroup{g}$ are both isomorphic to $\ZZ / p \ZZ$ so $G \iso \ZZ / p
\ZZ \times \ZZ / p \ZZ$.
\end{proof}
