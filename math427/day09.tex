\section{Day 9}

\subsection{Permutation Groups}

Recall that $\aut(X) = \lbrace \varphi : X \biject X \rbrace$.

\begin{df}
The \textbf{permutation group} $S_n$ is defined as
\[ S_n = \aut(\lbrace 1, 2, \dots, n \rbrace). \]
\end{df}

\begin{rem}
\[ |S_n| = n!. \]
\end{rem}

\begin{proof}
Let $\sigma \in S_n$. There are $n$ choices for $\sigma(1)$, $n - 1$
choices for $\sigma(2)$, and so on.
\end{proof}

We can notate permutations as
\[ \sigma = \begin{pmatrix}
1 & 2 & \cdots & n \\
\sigma(1) & \sigma(2) & \cdots & \sigma(n) \\
\end{pmatrix} \]

We can also use a slightly less verbose version called \textbf{cycle
notation}, in which we write $\sigma$ as a product of cycles $a_1 a_2
\cdots a_n$ in which $a_1 \xrightarrow{\sigma} a_2 \xrightarrow{\sigma}
\cdots \xrightarrow{\sigma} a_n \xrightarrow{\sigma} a_1$. For example,
we can write
\[ \sigma = \begin{pmatrix}
1 & 2 & 3 & 4 & 5 & 6 & 7 \\
4 & 6 & 3 & 2 & 7 & 1 & 5 \\
\end{pmatrix} = (1426)(3)(57) \in S_7. \]

We can view each independent cycle as a seperate permutation in $S_n$
that permutes the denoted elements but fixes everything else. i.e.
\[ S_7 \ni (1234) = \begin{pmatrix}
1 & 2 & 3 & 4 & 5 & 6 & 7 \\
2 & 3 & 4 & 1 & 5 & 6 & 7 \\
\end{pmatrix} \]
and we can view the multiplication as the composition in $S_n$.

\begin{df}
Above, we used cycle without defining what a cycle is. A permutation
$\sigma \in S_n$ is a \textbf{cycle} of length $r \leq n$ if $\exists$
distinct $x_1, x_2, \dots, x_r \in \lbrace 1, 2, \dots, n \rbrace$.
\begin{enumerate}
\item $\sigma(x_i) = x_{i + 1}$
\item $\sigma(x) = x$ for all $x \neq x_i$.
\end{enumerate}
This leads to the natural definition of cycle notation stated above.
\end{df}

\begin{df}
Let $\sigma \in S_n$. We say that $x$ is \textbf{fixed} by $\sigma$ if
$\sigma(x) = x$. Otherwise $x$ is \textbf{moved}.
\end{df}

\begin{df}
Two permutations $\sigma, \tau \in S_n$ are said to be \textbf{disjoint}
if for all $x \in \lbrace 1, 2, \dots, n \rbrace$,
\begin{itemize}
\item $x$ moved by $\sigma$ $\implies$ $x$ fixed by $\tau$.
\item $x$ moved by $\tau$ $\implies$ $x$ fixed by $\sigma$.
\end{itemize}
Notice here we don't care if both $\sigma, \tau$ fix the same elemnts.
They just can't move the same elements.
\end{df}

\begin{lem}
Let $\sigma, \tau \in S_n$ be two disjoint permutations (different from
cycles!). Then $\sigma \circ \tau = \tau \circ \sigma$.
\end{lem}

\begin{proof}
We have three cases.
\begin{enumerate}
\item $x$ is fixed by $\sigma$ and $\tau$.
\item $x$ is moved by $\sigma$.
\item $x$ is moved by $\tau$.
\end{enumerate}
Case 1 is trivial as $\sigma(\tau(x)) = x = \tau(\sigma(x))$.

For case 2, let $y = \sigma(x) \neq x$. We have $\sigma(y) =
\sigma(\sigma(x))$. If $\sigma(y) = \sigma(x) = y$, then $\sigma$ is not
a bijection so we have $\sigma(y) \neq y$, or $y$ is moved by $\sigma$,
so $y$ is fixed in $\tau$. $x$ is also fixed in $\tau$ because $x$ is
moved by $\sigma$, so we have $\tau(\sigma(x)) = \sigma(x) =
\sigma(\tau(x))$.

Case 3 is exactly the same as the above but with $\sigma$ and $\tau$
interchanged.
\end{proof}
