\documentclass{article}

\usepackage[margin=1in]{geometry}
\usepackage{amsmath}
\usepackage{amsthm}
\usepackage{amsfonts}
\usepackage{amssymb}
\usepackage{fancyhdr}
\usepackage{parskip}

% setup the header
\pagestyle{fancy}
\lhead{Will Song}
\chead{Math 427}
\rhead{HW1}

% setup definition/theorem etc
\newtheoremstyle{norm}
{3pt}
{3pt}
{}
{}
{\bf}
{:}
{.5em}
{}

\theoremstyle{norm}
\newtheorem{thm}{Theorem}[section]
\newtheorem{lem}[thm]{Lemma}
\newtheorem{df}[thm]{Definition}
\newtheorem{rem}[thm]{Remark}
\newtheorem{st}{Step}
\newtheorem{prop}[thm]{Proposition}
\newtheorem{cor}[thm]{Corollary}
\newtheorem{conj}[thm]{Conjecture}
\newtheorem{clm}[thm]{Claim}
\newtheorem{exr}[thm]{Exercise}
\newtheorem{ex}[thm]{Example}
\newtheorem{prb}[thm]{Problem}

% just useful shorthand
\renewcommand{\st}{\,\operatorname{s.t.}\,}
\let\hom\relax
\DeclareMathOperator{\hom}{Hom}
\DeclareMathOperator{\Tr}{Tr}
\DeclareMathOperator{\sgn}{sgn}
\DeclareMathOperator{\adj}{adj}

% for easier math
\everymath{\displaystyle}

\begin{document}

\begin{enumerate}
\item \begin{enumerate}
\item $n = 1 \cdot n + 0$
\item $n = q_1 m, m = q_2 d \implies n = (q_1 q_2) d$
\item $m = q_1 d, n = q_2 d \implies xm + yn = (xq_1 + yq_2) d$
\end{enumerate}

\item Consider the mapping $\phi : \mathbb{Z} \to
\mathbb{Z}/8\mathbb{Z}$ given by $n \mapsto \left[n^2\right]$. We then
have $\lbrace 1, 3, 5, 7 \rbrace \xrightarrow{\phi} \lbrace 1, 1, 1, 1
\rbrace$, which gives the desired result.

We could also write $(2n + 1)^2 = 4n^2 + 4n + 1 = 4n(n + 1) + 1$ which
is equivalent to $\frac{n}{2}(n + 1) \cdot 8 + 1$ or $\frac{n + 1}{2} n
\cdot 8 + 1$, which works because one of $\frac{n}{2}$ and $\frac{n +
1}{2}$ is integral.

\item Consider the mapping $\gamma : \lbrace 0, 1, 2 \rbrace \to \lbrace
0, 1, 2 \rbrace^3$ given by
\[ \begin{aligned}
0 &\mapsto (0, 1, 2) \\
1 &\mapsto (1, 2, 0) \\
2 &\mapsto (2, 0, 1) \\
\end{aligned} \]
and the natural injection $\pi : \mathbb{Z} \to \lbrace 0, 1, 2 \rbrace$
given by $3q + r \mapsto r$ with $0 \leq r < 3$. Naturally, $\gamma
\circ \pi$ becomes the injection from $n$ to the residues modulo $3$ of
the $3$ consecutive integers starting fron $n$, and $0$ is always one of
them.

\item We take the canonical definition of a prime in an integral domain
to be that $p$ is considered prime if and only if $k \nmid p$, $\forall
k \in \mathbb{Z}^+ \backslash \lbrace 1, p \rbrace$.

For the forwards direction, assume the contrary and say that $p \mid mn
\implies p \nmid m, p \nmid n$. Then $m = q_1 p + r_1, n = q_2 p + r_2
\implies qp + 0 = mn = (q_1 q_2) p^2 + (q_1 r_2 + q_2 r_1) p + r_1 r_2 =
(q_1 q_2 p + q_1 r_2 + q_2 r_1) p + r_1 r_2 \implies r_1 r_2 = 0$, but
$r_1, r_2 > 0$ so we arrive at a contradiction.

For the backwards direction, again assume the contrary and claim that $k
\mid p$ for some $k \neq 1, p$.  We wish to contradict the statement
that one of $p \mid m, p \mid n$ must be true, but this is easy because
$p = jk$ and the construction $m = j, n = k$ clearly works as $p > j, k$
due to the silly fact that $a \nmid b$ if $a > b > 0$.

\item The backwards direction is true by definition (i.e. $\gcd(a, b) =
\max \lbrace n : n \mid a, n \mid b \rbrace$). If $m \mid n$, then $m
\leq n$ and $\gcd(m, n) \geq m$ ($m \mid m, m \mid n$), but $\gcd(a, b)
\leq \min(a, b)$ because (again) of the fact that $a \nmid b$ if $a > b
> 0$, so $\gcd(m, n \leq m$ as well, implying $\gcd(m, n) = m$.

\item Let $d = \gcd(m, n)$. $d \mid m$ so $d \mid m_1$ and $d \mid n$ so
$d \mid n_1$, so $d \in S = \lbrace k : k \mid m_1, k \mid n_1 \rbrace$.
If $s \in S \implies s \mid \max S$, we are done. We then have, by
Bezout,
\[ \gcd(m_1, n_1) = \max S = a m_1 + b n_1 = a q_1 d + b q_2 d = (a q_1
+ b q_2) d, \]
which gives the desired result.
\end{enumerate}

\end{document}
