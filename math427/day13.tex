\section{September 23, 2015}

\subsection{More Cosets}

\begin{prop}
Let $H \leq G$.
\[ H\,\textrm{normal} \iff aH = Ha. \]
\end{prop}

\begin{proof}
If $H$ is normal, then
\[ aH = \lbrace ah : h \in H \rbrace = \lbrace aa^{-1}ha : h \in H
\rbrace = \lbrace ha : h \in H \rbrace \subseteq Ha. \]
Similarly, $Ha \subseteq aH$ so $aH = Ha$.

If $aH = Ha$, then $ah_1 = h_2a$, or $ah_1a^{-1} = h_2 \in H$, so $H$ is
normal.
\end{proof}

\begin{cor}
Let $H \leq G$ and let $\lvert G / H \rvert = 2$. Then $H$ is normal.
\end{cor}

\begin{proof}
$G/H$ is composed of $H$ and $G \setminus H$. $H \backslash G$ is
composed of $H$ and $G \setminus H$ as well. The left cosets are the
right cosets.
\end{proof}

\begin{thm}
Let $N \lhd G$ be a normal subgroup of $G$. Then the set of cosets $G /
N$ is a group under the standard multiplication $(aN)(bN) = (ab)N$ and
the projection map $\pi : G \surj G / N$ a homomorphism.
\end{thm}

\begin{proof}
We confirm that multiplication is well defined. If $aN = a'N, bN = b'N$,
then $a' = an_1, b' = bn_2$, or $a'b' = an_1bn_2 = abb^{-1}n_1bn_2 =
abn_3n_2$. Confirming that $G/N$ is a group under this multiplication is
trivial. Then by construction, we get that $\pi$ is a surjective
homomorphism.
\end{proof}

\begin{thm}[First Isomorphism Theorem]
Let $\varphi : G \to H$ be a homomorphism. Then there is a unique
isomorphism $\overline{\varphi} : G / \ker\varphi \to \varphi(G)$ such
that the following square commutes.
\begin{center}
\begin{tikzpicture}[scale=3, nodes={scale=1.2}]
\node (G)   at (0, 1) {$G$};
\node (H)   at (2, 1) {$H$};
\node (ker) at (0, 0) {$G / \ker\varphi$};
\node (im)  at (2, 0) {$\varphi(G)$};

\path[->] (G)   edge         node[above]{$\varphi$}            (H)
          (G)   edge         node[left] {$\pi$}                (ker)
          (ker) edge[dashed] node[below]{$\overline{\varphi}$} (im);
\path[right hook->] (im) edge node[right]{$\iota$} (H);
\end{tikzpicture}
\end{center}
or in other words, $\overline{\varphi}(aN) = \varphi(a)$.

This is also called the homomorphism theorem.
\end{thm}

\begin{proof}
Let $N = \ker\varphi$.
\begin{enumerate}
\item We first verify that $\overline\varphi$ is well defined, which is
easy because if $aN = bN$, then $b = an$ for some $n \in \ker\varphi$ so
\[ \overline\varphi(bN) = \varphi(b) = \varphi(an) =
 \varphi(a)\varphi(n) = \varphi(a) = \overline\varphi(aN). \]
\item $\overline\varphi$ is a homomorphism because
\[ \overline\varphi((aN)(bN)) = \overline\varphi((ab)N) = \varphi(ab) =
\varphi(a)\varphi(b) = \overline\varphi(aN) \overline\varphi(bN). \]
\item $\overline\varphi$ is clearly surjective so we verify injectivity
by computing the kernel.
\[ \ker\overline\varphi = \lbrace aN : \varphi(a) = e_H \rbrace =
\lbrace aH : a \in N \rbrace = \lbrace N \rbrace = eH. \]
\end{enumerate}
\end{proof}

\begin{ex}
Let $G$ be some group and let $g \in G$. Define the homomorphism
$\varphi : \ZZ \to G$ by $n \mapsto g^n$. Clearly $\varphi(\ZZ) =
\cycgroup{g}$ and $\ker\varphi = n\ZZ$ for some $n \in \ZZ$. Then we
by the first isomorphism theorem that there is a well defined
isomorphism $\overline\varphi : \ZZ / n\ZZ \to \cycgroup{g}$ given by
$k + n\ZZ \mapsto g^k$.
\end{ex}
