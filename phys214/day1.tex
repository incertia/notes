\section{Day 1}

Two basic types of waves, \textbf{longitudinal waves} and
\textbf{transverse waves}.

Waves have a few identifying properties:
\begin{itemize}
\item Wavenumber $k$
\item Wavelength $\lambda$
\item Angular frequency $\omega$
\item Frequency $f$
\item Period $T$
\item Amplitude $A$
\item Position, time coordinates $(x, t)$
\end{itemize}

Waves can be represented as
\[ y = A \cos (kx - \omega t), \]
where $k = \frac{2 \pi}{\lambda}$ and $\omega = 2 \pi f$.

The velocity of a wave is simply $v = \frac{\lambda}{T}$.

We don't usually measure the amplitude of a wave, but measure the
intensity (power transmitted / area), which happens to be proportional
to $A^2$.

Waves originating from a point source move radially.

\begin{rem}[Key fact from differential equations]
If $\Psi_1$ and $\Psi_2$ are both solutions to a differential equation,
then $\Psi_1 + \Psi_2$ is also a solution.
\end{rem}

We can add on to this with wave behavior to get the result of a
composition of two waves.
